
\section*{Exercise 5}

A good idea to think how to construct the terms is to reason over the
types that they should have. Even if the system is untyped, this
strategy works. Type systems usually restrict which terms can be constructed,
but they do not affect reduction. If we construct a term assuming a type
system and it reduces as we want, it will work in the untyped calculus.

In this exersise whe should encode the universal property of coproduct.
$\Sigma_i$ are the constructors (usually $inl$, $inr$ in functional
programming literature), so they have type $A_i \rightarrow A_1 + A_2$.

Since we do not have a primitive way to construct a sum type (in the way
we defined our calculus), we have to encode it in
the same way we encode, for example Booleans with the type
$ X \rightarrow X \rightarrow X$ (given two terms of type $X$,
in which one you choose you have a bool of information).

Using this reasoning, a good way to type the terms is:
$$\Sigma_i : A_i \rightarrow
            (A_1 \rightarrow B) \rightarrow
            (A_2 \rightarrow B) \rightarrow
            B
$$


So
$$
    \Sigma_1 = \lambda t f g . f t \\
$$
$$
    \Sigma_2 = \lambda t f g . g t
$$

are good definitions.

$C$ is the destructor of this case analysis, it should have type

$$
    C : A_1 + A_2 \rightarrow
       (A_1 \rightarrow B) \rightarrow
       (A_2 \rightarrow B) \rightarrow
       B
$$

So
$$
     C = \lambda c f g . c f g \\
$$

is a possible definition (recall: unfold the definition of $A_1 + A_2$
and this type checks).

Finally we can verify:

$$
    \underline{C} (\Sigma_1 t) f_1 f_2
    \succ^{3}  (\Sigma_1 t) f_1 f_2
    = (\underline{\lambda t f g} . f t) t f_1 f_2
    \succ^{3} f_1 t
$$
$$
    \underline{C} (\Sigma_2 t) f_1 f_2
    \succ^{3}  (\Sigma_2 t) f_1 f_2
    = (\underline{\lambda t f g} . g t) t f_1 f_2
    \succ^{3} f_2 t
$$

note: Actually $C = Id$ works fine and we need less reductions, I preferred
this presentation because is inmediate from some program calculation from
types, I think also that this technique will be useful to derive
harder $\lambda$-programs.

note: This is probably not the unique way to code coproducts.

note: $C$ is actually a (prefix) {\tt case} construction,
($C \: u \: f \: g$) actually models:

{\tt
  case u of\\
  {\setlength\parindent{24pt}
    \indent inl t  $\rightarrow$ f t\\
    \indent inr t  $\rightarrow$ g t
  }
}
