\newcommand{\NatF}
           {\forall X. (X \Rightarrow X) \Rightarrow X \Rightarrow X}

%\newcommand{\N}
%           {\mathfrak{N}}
           
\newcommand{\Arr}
           {\Rightarrow}

           
\section{Exercise 37 - Arithmetic functions in SystemF}

One way to solve this exercise is to do in the same way then in Exercise
14, constructing a proof tree and collecting equations
to compose a system and then find a unifier.

This could work by hand in this small examples, as a guide to figure out
types, but not as a general rule as SystemF typing is not decidable.
Proofs are no longer syntax directed and unification is no longer first
order.

\footnote{We saw in the course an interesting way to
  (if I understood it well) move all complexity to the equations
  doing subtyping, which at least greatly improves error messages in
  an implementation when the algorithm stucks}

Since we want to write {\bf a} type, and it is very intuitive which one
we want for this cases, We will show a typing derivation for that
type instead of trying to infer one from scratch.
To define arithmetic operations we use the very well known terms, we will not
prove here that they compute the desired value.

Remark:
Note that each time we apply a forall introduction, the binded variable does
not occurs free in the context, we will not point this every time.


We will use the notation
$\N := \NatF$, folding and unfolding this definition when we
need it.

\subsection*{Addition}


\begin{theorem}
  Let $Add = \lambda m n f x . m f (n f x)$, then:
  $\emptyset \vdash Add : \N \Arr \N \Arr \N$
\end{theorem}

\newenvironment{scprooftree}[1]%
  {\gdef\scalefactor{#1}\begin{center}\proofSkipAmount \leavevmode}%
  {\scalebox{\scalefactor}{\DisplayProof}\proofSkipAmount \end{center} }


\begin{proof}
  We construct a proof tree:

  
  \begin{scprooftree}{0.80}
    \def\fCenter{\vdash}

    \AxiomC{}
    \RightLabel{Ax}
    \UnaryInfC {$\Gamma \fCenter\ m : \NatF$}
    \RightLabel{$\forall_e$}
    \UnaryInfC{$\Gamma 
      \fCenter\ m : (X\Arr X)\Arr X \Arr X$}

    \AxiomC{}
    \RightLabel{Ax}
    \UnaryInfC{$\Gamma \fCenter f : X\Arr X$}
    \RightLabel{$\Arr_e$}
    \BinaryInfC{$ \Gamma \fCenter\ m f : X \Arr X $}
    

    %----
    \AxiomC{}
    \RightLabel{Ax}
    \UnaryInfC{$\Gamma \fCenter n : \forall X. (X \Arr X)\Arr X \Arr X$}
    \RightLabel{$\forall_e$}
    \UnaryInfC{$\Gamma \fCenter n : (X \Arr X)\Arr X \Arr X$}
    \AxiomC{}
    \RightLabel{Ax}
    \UnaryInfC{$\Gamma \fCenter\ f : X\Arr X$}
    \RightLabel{$\Arr_e$}
    \BinaryInfC{$\Gamma \fCenter\ n f : X\Arr X$}

    \AxiomC{}
    \RightLabel{Ax}
    \UnaryInfC{$\Gamma \fCenter\ x : X$}
    \RightLabel{$\Arr_e$}
    \BinaryInfC{$ \Gamma \fCenter\ n f x : X $}
                                              \RightLabel{$\Arr_e$}
    \BinaryInfC{$ (\Gamma \equiv) m,n:\N, f: X\Rightarrow X, x:X
                 \fCenter\ m f (n f x) : X $}
                                              \RightLabel{$\Rightarrow_i$}
    
    \UnaryInfC{$ m,n:\N, f: X\Rightarrow X \fCenter\ \lambda x. m f (n f x)
      : \Rightarrow X \Rightarrow X $}
                                              \RightLabel{$\Rightarrow_i$}
    
    \UnaryInfC{$ m,n:\N \fCenter\ \lambda f x. m f (n f x)
      : (X\Rightarrow X) \Rightarrow X \Rightarrow X $}
                                              \RightLabel{$\forall_i$}
    
    \UnaryInfC{$ m,n:\N \fCenter\ \lambda f x. m f (n f x) :\NatF(\equiv\N)$}
                                              \RightLabel{$\Rightarrow_i$}
    \UnaryInfC{$ m:\N \fCenter\ \lambda n f x. m f (n f x) : \N\Arr\N$}
                                              \RightLabel{$\Rightarrow_i$}
    \UnaryInfC{$\vdash \lambda m\, n\, f\, x. m f (n f x) : \N \Arr \N \Arr \N$}
    
  \end{scprooftree}
\end{proof}



\subsection*{Mutiplication}

\begin{theorem}
  Let $Mult = \lambda m n f x . m (n f) x$, then:
  $\emptyset \vdash Mult : \N \Arr \N \Arr \N$
\end{theorem}
\begin{proof}
  
  Again, we construct the derivation. To have a smaller proof we apply
  multiple elimination rules in one step (the usual intros tactic).
  
  \begin{scprooftree}{1.0}
    \def\fCenter{\vdash}

    \AxiomC{}
    \RightLabel{Ax}
    \UnaryInfC{$ \Gamma \fCenter m : \forall X. (X\Arr X) \Arr X \Arr X$}
    \RightLabel{$\forall_i$}
    \UnaryInfC{$ \Gamma \fCenter m : (X\Arr X) \Arr X \Arr X$}

    \AxiomC{}
    \RightLabel{Ax}
    \UnaryInfC{$\Gamma \fCenter n : \forall X. (X \Arr X) \Arr X \Arr X$}
    \RightLabel{$\forall_i$}
    \UnaryInfC{$\Gamma \fCenter n : (X \Arr X) \Arr X \Arr X$}

    \AxiomC{}
    \RightLabel{Ax}
    \UnaryInfC{$\Gamma \fCenter f : X \Arr X$}
    \RightLabel{$\Arr_e$}
    \BinaryInfC{$ \Gamma \fCenter (n f) : X\Arr X$}
    \RightLabel{$\Arr_e$}
    \BinaryInfC{$\Gamma \fCenter m (n f): X \Arr X$}

    \AxiomC{}
    
    \RightLabel{Ax}
    \UnaryInfC{$\Gamma \fCenter x : X$}
    \RightLabel{$\Arr_e$}
    \BinaryInfC{$(\Gamma :=) m,n:\N,f:X\Arr X,x:X \fCenter m(n f) x :X$}
    \RightLabel{$\Arr_i,\Arr_i,\forall_i,\Arr_i,\Arr_i$}
    \UnaryInfC{$\fCenter \lambda m n f x . m (n f) x : \N\Arr\N\Arr\N$}
  \end{scprooftree}
\end{proof}



\subsection*{Exponentiation}


\begin{theorem}
  Let $Exp = \lambda m n. n m $, then:
  $\emptyset \vdash Exp : \N \Arr \N \Arr \N$
\end{theorem}
\begin{proof}
  
  This is by far the most interesting example. Recall that this function
  did not have this desired type in simply typed $\lambda$-calculus.
  Now it is possible to type it as we wish. This shows the power of
  universal quantification. Note that when we eliminate the $\forall$
  quantifier (reading the proof bottom up), the instantiation is not
  syntax directed, we guess a type on the left branch.
  
  \begin{scprooftree}{0.9}
    \def\fCenter{\vdash}

    \AxiomC{}
    \RightLabel{Ax}
    \UnaryInfC{$m,n:\N \fCenter n: \N$}
    \RightLabel{$\forall_e$ $[X/X\Arr X]$}
    \UnaryInfC{$m,n:\N \fCenter n:
      ((X\Arr X)\Arr(X\Arr X))\Arr (X\Arr X)\Arr(X\Arr X)$}

    \AxiomC{}
    \RightLabel{Ax}
    \UnaryInfC{$m,n:\N \fCenter m: \N$}
    \RightLabel{$\forall_e$ $[X/X]$}
    \UnaryInfC{$m,n:\N \fCenter m: (X\Arr X)\Arr(X\Arr X)$}
    %\BinaryInfC{$m,n:\N \fCenter n m
    %           : ((X\Arr X)\Arr(X\Arr X))\Arr (X\Arr X)\Arr(X\Arr X)$}

    
    \BinaryInfC{$m,n:\N \fCenter n m
               : (X\Arr X)\Arr(X\Arr X)$}
    \RightLabel{$\forall_i$}
    \UnaryInfC{$m,n:\N \fCenter n m :\NatF$}
    \RightLabel{$\Arr_i, \Arr_i$}
    \UnaryInfC{$ \fCenter \lambda m n . n m : \N \Arr \N \Arr \N$}

  \end{scprooftree}
\end{proof}

%TODO: Note that this blah blah
