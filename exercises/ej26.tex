\section{Exercise 26}

Again we will not show a strict formal proof, but they can be implemented
trivially. We assume implicitly that we previously derived some usual rules
such as $RAA$ or $\iff_{intro}$. So to prove equivalence we prove
double implication. Also in the second equivalence we make a proof
by contradiction, and use $\neg \exists_s x. \neg A \iff \forall_s x . A $.

\begin{theorem}
  $A[x:=\epsilon_s (\lambda x. A)] \Leftrightarrow \exists_s x . A$
\end{theorem}
\begin{proof}

  The left to right implication is trivial applying the rule $\exists_i$
  of the exercise 24. We only need that
  $\gamma \vdash \epsilon_s (\lambda x. A) : s$ which is well typed.

  For the opposite we apply the $\exists_e$ rule of the exercise 24.

\end{proof}


\begin{theorem}
  $A[x:=\epsilon_s (\lambda x. \neg A)] \Leftrightarrow \forall_s x . A$
\end{theorem}
\begin{proof}
  We are using classical logic so assume by contradiction
  $\exists_s x . \neg A$. Then applying $\exists_e$ with
  $\neg A \Rightarrow \neg A [x:=\epsilon_s (\lambda x. \neg A)]$
  (which is provable) we prove $\neg A [x:=\epsilon_s (\lambda x. \neg A)]$.
  By contraposition, we proved
  $A[x:=\epsilon_s (\lambda x. \neg A)] \Rightarrow \forall_s x . A$ (using
  De Morgan's law and double negation elimination).

  The opposite is trivial, apply the $\epsilon$ rule.
\end{proof}
