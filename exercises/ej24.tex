\section{Exercise 24}

Let's start for what we can prove.
The second rule is ok even without weakening.

\begin{theorem}

  The following rule is derivable, assuming $x \notin FV(B)$:
  
  
  \begin{prooftree}
    \AxiomC{$\gamma ; \Gamma \vdash \exists_s x . A$}
    \AxiomC{$\gamma, x:s ; \Gamma \vdash A \Rightarrow B$}
       \RightLabel{$\exists_e$}
    \BinaryInfC{$\gamma; \Gamma \vdash B$}
  \end{prooftree}

\end{theorem}


% some definitions

\newcommand{\existsX}{\exists_s x}
\newcommand{\existsXUnfolded}
           {\lambda X. \forall_o Y .
             (\forall_s x. X x \Rightarrow Y) \Rightarrow Y }

\newcommand{\existsApplied}
           {\forall_o Y .
             (\forall_s x. A x \Rightarrow Y)
             \Rightarrow Y }

           
\begin{proof}

  Note that $\existsX A \equiv_{\text{def}}
  (\existsXUnfolded) A \succ_{\beta} \existsApplied $

\begin{prooftree}

  \AxiomC{$\gamma, x:s ; \Gamma \vdash A \Rightarrow B$}
     \RightLabel{$\forall_i$}
  \UnaryInfC{$\gamma ; \Gamma \vdash \forall x_s . (A x \Rightarrow B)$}
  
  \AxiomC{$\gamma \vdash B : o$}
  
  \AxiomC{$\gamma \vdash \existsX A : o $}
  \AxiomC{$  \existsApplied \equiv_{\beta}  \existsX A$}
  \AxiomC{$\gamma; \Gamma \vdash \existsX A$}
     \RightLabel{$\beta$}
  \TrinaryInfC{$\gamma; \Gamma \vdash \existsApplied $}

     \RightLabel{$\forall_e$}
  \BinaryInfC{$ \gamma ; \Gamma \vdash
    (\forall_s x. A x \Rightarrow B) \Rightarrow B$}

     \RightLabel{$\Rightarrow_e$}
  \BinaryInfC{$\gamma ; \Gamma \vdash B$}
\end{prooftree}

Note that leaves are axioms, a $\beta$-equivalence,
and $\gamma \vdash B:o$ which must be true since
$ \gamma, x:s \vdash (A \Rightarrow B) : o$


\end{proof}


To derive a left rule, we need to manipulate context:

\begin{theorem}[Left $\exists$-elimination]
  
  The following rule is admissible:
  

  \begin{prooftree}
    \AxiomC{$\gamma;\Gamma, \exists_s x . A \vdash B$}
    \RightLabel{$L\exists_{e}$}
    \UnaryInfC{$\gamma;\Gamma, A[x:=e] \vdash B$}
  \end{prooftree}

\end{theorem}

\begin{proof}

  We write a derivation, note that leaves are premises. Also note that we use
  weakening and a rule that was not derivable without it ($\exists_i$). So
  we are proving admissibility:

  \begin{prooftree}
    
    \AxiomC{$ \gamma;\Gamma, \exists_s x . A \vdash B$}
    \RightLabel{$\forall_i$}
    \UnaryInfC{$\gamma; \Gamma \vdash \exists_s x . A \Rightarrow B $}
    \RightLabel{$wk$}
    \UnaryInfC{$\gamma; \Gamma, A[x:=e] \vdash \exists_s x . A \Rightarrow B $}
    \AxiomC{$\gamma \vdash e : s$}
    \AxiomC{}
    \RightLabel{Ax}
    \UnaryInfC{$\gamma, \Gamma, A[x:=e]\vdash A [x:=e] $}
    \RightLabel{$\exists_i$}
    \BinaryInfC{$\gamma;\Gamma, A[x:=e] \vdash \exists_s x . A$}

    \RightLabel{$\Rightarrow_e$}
    \BinaryInfC{$\gamma;\Gamma, A[x:=e] \vdash B$}
  \end{prooftree}

\end{proof}

The first rule given is also admissible. Again, now that we showed some formal
proofs, let's be more sloppy here.
(And also because actually we will be using
admissible but not derivable rules everywhere, to be really formal we need
some boring work before).

We want to prove $\gamma; \Gamma \vdash \exists_s x . A$, if we unfold the
definition of $\exists$ and do a beta reduction on the goal this is
$\gamma; \Gamma \vdash
\forall Y. (\forall_s x. A(x) \Rightarrow Y) \Rightarrow Y $.
Introducing the universal quantifier and the first paremeter of the arrow,
the goal is
$\gamma; \Gamma, Y:o, \forall_s x. A(x) \Rightarrow Y \vdash Y $.
Since we can derive $A[x:=e]$, applying
$\forall_s x. A(x) \Rightarrow Y$ to $e$ and $A[x:=e]$ we have what we want.

Note that we are using a lot of non-primitive rules here. 

