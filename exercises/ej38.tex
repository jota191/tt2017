\section{Exercise 38 - Product and Coproduct in System F}

%\newcommand{\Arr}{\Rightarrow}

Terms are the same as in simply typed lambda calculus.
Again we will ommit computing reductions (obviowsly we must prove that
this terms reduces to what we want) since this was
well studied in class and at IMERL's hall, in previous exercises,
(and actually it is very easy in most cases).

Type derivations, in the other hand, will be provided.


\subsection*{Product}

Let:

\begin{align*}
  A \times B := \forall X. (A\Arr B\Arr X) \Arr X \nonumber
\end{align*}
  
  The type of the -unique- constructor $pair$, and destructors $\pi_1$ and
  $\pi_2$ are:

\begin{align}
  pair  :& A \Arr B \Arr A\times B  \nonumber \\
  \pi_1 :& A \times B \Arr A        \nonumber \\
  \pi_2 :& A \times B \Arr B        \nonumber
\end{align}
  
  where they are defined as:

\begin{align}
  pair :=& \lambda a b p . p a b        \nonumber \\
  \pi_1:=& \lambda p. p (\lambda t f.t) \nonumber \\
  \pi_2:=& \lambda p. p (\lambda t f.f) \nonumber
\end{align}

We do not use actual booleans since they would not be well typed here.

Denoting $pair\ a\ b$ as $(a,b)$ it is easy to show that:

\begin{align*}
  \pi_1 (a,b) &\succ^{*} a \\
  \pi_2 (a,b) &\succ^{*} b
\end{align*}

The typing proofs are the following:

For $\pi_1$ (for $\pi_2$ is analogous):

\newcommand{\AxB}{ \forall X. (A\Arr B \Arr X) \Arr X }

\begin{scprooftree}{0.95}
  \def\fCenter{\vdash}

  \AxiomC{}
  \RightLabel{Ax}
  \UnaryInfC{$p : \AxB \fCenter p : \forall X. (A\Arr B \Arr X) \Arr X$}
  \RightLabel{$\forall_e$ $[X/A]$}
  \UnaryInfC{$p : \AxB \fCenter p : (A\Arr B \Arr A) \Arr A$}

  \AxiomC{}
  \RightLabel{Ax}
  \UnaryInfC{$p : \AxB, t:A, f:B \fCenter t :A $}
  \RightLabel{$\Arr_i, \Arr_i$}
  \UnaryInfC{$p : \AxB \fCenter  (\lambda t f. t) : (A\Arr B \Arr A)$}
  \BinaryInfC{$ p : \forall X. (A\Arr B \Arr X) \Arr X
    \fCenter p (\lambda t f. t) : A$}
  \RightLabel{$\Arr_i$}
  \UnaryInfC{$\fCenter \lambda p.p (\lambda t f. t) : A\times B\Arr A$}
\end{scprooftree}


\newpage
For $pair$:

\begin{scprooftree}{0.95}
  \def\fCenter{\vdash}

  \AxiomC{}
  \RightLabel{Ax}
  \UnaryInfC{$\Gamma \fCenter p : A \Arr B \Arr X$}
  
  \AxiomC{}
  \RightLabel{Ax}
  \UnaryInfC{$\Gamma \fCenter a : A$}
  \RightLabel{$\Arr_e$}
  \BinaryInfC{$\Gamma \fCenter p \, a : B \Arr X$}
  
  \AxiomC{}
  \RightLabel{Ax}
  \UnaryInfC{$ \Gamma \fCenter b : B$}
  \RightLabel{$\Arr_e$}
  \BinaryInfC{$(\Gamma := )a : A, b:B, p : A \Arr B \Arr X \fCenter p a b : X$}
  \RightLabel{$\Arr_i, \Arr_i, \forall_i, \Arr_i$}
  \UnaryInfC{$ \fCenter \lambda a b p . p a b : A \Arr B \Arr A \times B $}
\end{scprooftree}


\subsection*{Co-Product}

Let:

\begin{align}
  A + B := \forall X. (A\Arr X) \Arr (B\Arr X) \Arr X \nonumber
\end{align}

Now we have two constructors, $inl$, $inr$
and the case analysis destructor $C$:

\begin{align*}
  inl :& A  \Arr A + B   \\
  inr :& B  \Arr A + B  \\
  C   :& A + B \Arr (A \Arr C) \Arr (B \Arr C) \Arr C
\end{align*}

where they are defined as:

\begin{align*}
  inl :=& \lambda a l r . l a        \\
  inr :=& \lambda b l r . r b        \\
  C   :=& \lambda x . x
\end{align*}

Note that we chosed the simple implementation of C now,
in contrast to Exo 5, (but of course $C = \lambda c f g . c f g$ typechecks
also, they are $\eta$-equivalent).

We provide typing proofs for $inl$ and for $C$ ($inr$ is analogous to $inl$)

\begin{scprooftree}{0.95}
  \def\fCenter{\vdash}

  \AxiomC{}
  \RightLabel{Ax}
  \UnaryInfC{$ \Gamma \fCenter l : A \Arr X$}

  \AxiomC{}
  \RightLabel{Ax}
  \UnaryInfC{$ \Gamma \fCenter a : A$}
  \RightLabel{$\Arr_e$}
  \BinaryInfC{$ (\Gamma :=) a : A, l : A \Arr X, r : B \Arr X \fCenter l a : X$}
  \RightLabel{$\Arr_i, \forall_i, \Arr_i, \Arr_i$}
  \UnaryInfC{$ \fCenter \lambda a l r . l a
    : A \Arr \forall X. (A \Arr X) \Arr (B \Arr X) \Arr X $}
\end{scprooftree}


\begin{scprooftree}{0.95}
  \def\fCenter{\vdash}

  \AxiomC{}
  \RightLabel{Ax}
  \UnaryInfC{$   x : (\forall X. (A \Arr X) \Arr (B \Arr X) \Arr X
    \fCenter x: (\forall X. (A \Arr X) \Arr (B \Arr X) \Arr X$}
  \RightLabel{$\forall_e$ $[X/C]$}
  \UnaryInfC{$ x : (\forall X. (A \Arr X) \Arr (B \Arr X) \Arr X)
    \fCenter x : (A \Arr C) \Arr (B \Arr C) \Arr C$}
  \RightLabel{$\Arr_i$}
  \UnaryInfC{$\fCenter \lambda x . x
    : (\forall X. (A \Arr X) \Arr (B \Arr X) \Arr X)
    \Arr (A \Arr C) \Arr (B \Arr C) \Arr C $}
\end{scprooftree}

When we make the substitution $X := C$ in types we are actually assuming
a more general context where $C$ is a type, no problem.
