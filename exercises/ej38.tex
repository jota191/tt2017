\section{Exercise 38 - Product and Coproduct in System F}

%\newcommand{\Arr}{\Rightarrow}

Terms are the same as in simply typed lambda calculus.
We will ommit computing reductions (obviowsly we must prove that
this terms reduces to what we want) since this was
well studied in class and at IMERL's hall, and it is trivial.

Type derivations, in the other hand, will be provided if they are
non trivial.

%TODO:
%The algorithm to encode types in System F is straightforward. Let
%$T$ be a fresh variable, representing the type to encode, consider the type
%of constructors $C_i$ ($C_i$ is of the form )

\subsection*{Product}

\begin{align}
  A \times B := \forall X. (A\Arr B\Arr X) \Arr X \nonumber
\end{align}
  
  The type of the -unique- constructor $pair$, and destructors $\pi_1$ and
  $\pi_2$ are:

\begin{align}
  pair  :& A \Arr B \Arr A\times B  \nonumber \\
  \pi_1 :& A \times B \Arr A        \nonumber \\
  \pi_2 :& A \times B \Arr B        \nonumber
\end{align}
  
  where they are defined as:

\begin{align}
  pair :=& \lambda a b p . p a b        \nonumber \\
  \pi_1:=& \lambda p. p (\lambda t f.t) \nonumber \\
  \pi_2:=& \lambda p. p (\lambda t f.f) \nonumber
\end{align}

Denoting $pair\ a\ b$ as $(a,b)$ it is easy to show that:

\begin{align*}
  \pi_1 (a,b) &\succ^{*} a \\
  \pi_2 (a,b) &\succ^{*} b
\end{align*}

The typing proofs are the following:

For $\pi_1$ (for $\pi_2$ is analogous):

\newcommand{\AxB}{ \forall X. (A\Arr B \Arr X) \Arr X }

\begin{scprooftree}{0.95}
  \def\fCenter{\vdash}

  \AxiomC{}
  \RightLabel{Ax}
  \UnaryInfC{$p : \AxB \fCenter p : \forall X. (A\Arr B \Arr X) \Arr X$}
  \RightLabel{$\forall_e$}
  \UnaryInfC{$p : \AxB \fCenter p : (A\Arr B \Arr A) \Arr A$}
+
  \AxiomC{}
  \RightLabel{Ax}
  \UnaryInfC{$p : \AxB, t:A, f:B \fCenter t :A $}
  \RightLabel{$\Arr_i, \Arr_i$}
  \UnaryInfC{$p : \AxB \fCenter  (\lambda t f. t) : (A\Arr B \Arr A)$}
  \BinaryInfC{$ p : \forall X. (A\Arr B \Arr X) \Arr X
    \fCenter p (\lambda t f. t) : A$}
  \RightLabel{$\Arr_i$}
  \UnaryInfC{$\fCenter \lambda p.p (\lambda t f. t) : A\times B\Arr A$}
\end{scprooftree}

For $pair$:

\begin{scprooftree}{0.95}
  \def\fCenter{\vdash}

  \AxiomC{}
  \UnaryInfC{$\Gamma \fCenter p : A \Arr B \Arr X$}
  
  \AxiomC{}
  \UnaryInfC{$\Gamma \fCenter a : A$}
  \BinaryInfC{$\Gamma \fCenter p a : B \Arr X$}
  
  \AxiomC{}
  \RightLabel{Ax}
  \UnaryInfC{$ \Gamma \fCenter b : B$}
  \BinaryInfC{$(\Gamma := )a : A, b:B, p : A \Arr B \Arr X \fCenter p a b : X$}
  \RightLabel{$\Arr_i, \Arr_i, \forall_i, \Arr_i$}
  \UnaryInfC{$ \fCenter \lambda a b p . p a b : A \Arr B \Arr A \times B $}
\end{scprooftree}

\subsection*{Co-Product}

Let:

\begin{align}
  A + B := \forall X. (A\Arr X) \Arr (B\Arr X) \Arr X \nonumber
\end{align}

Now we have two constructors, $inl$, $inr$
and the case analysis destructor $C$:

\begin{align*}
  inl :& A  \Arr A + B   \\
  inr :& B  \Arr A + B  \\
  C   :& A + B \Arr (A \Arr C) \Arr (B \Arr C) \Arr C
\end{align*}

and the definitions:

  where they are defined as:

\begin{align*}
  inl :=& \lambda a l r . l a        \\
  inr :=& \lambda b l r . r b        \\
  C   :=& \lambda x . x
\end{align*}

Note that we opted for the simple implementation of C now,
in contrast to Exo 5, (but of course $C = \lambda c f g . c f g$ typechecks
also).

We provide a typing proof for $inl$ and for $C$ ($inr$ is analogous to $inl$)

\begin{scprooftree}{0.95}
  \def\fCenter{\vdash}

  \AxiomC{}
  \RightLabel{Ax}
  \UnaryInfC{$ \Gamma \fCenter l : A \Arr X$}

  \AxiomC{}
  \RightLabel{Ax}
  \UnaryInfC{$ \Gamma \fCenter a : A$}
  \BinaryInfC{$ (\Gamma :=) a : A, l : A \Arr X, r : B \Arr X \fCenter l a : X$}
  \RightLabel{$\Arr_i, \forall_i, \Arr_i, \Arr_i$}
  \UnaryInfC{$ \fCenter \lambda a l r . l a
    : A \Arr \forall X. (A \Arr X) \Arr (B \Arr X) \Arr X $}
\end{scprooftree}


\begin{scprooftree}{0.95}
  \def\fCenter{\vdash}

  \AxiomC{}
  \RightLabel{Ax}
  \UnaryInfC{$   x : (\forall X. (A \Arr X) \Arr (B \Arr X) \Arr X
    \fCenter x: (\forall X. (A \Arr X) \Arr (B \Arr X) \Arr X$}
  \RightLabel{$\forall_e$}
  \UnaryInfC{$ x : (\forall X. (A \Arr X) \Arr (B \Arr X) \Arr X)
    \fCenter x : (A \Arr C) \Arr (B \Arr C) \Arr C$}
  \RightLabel{$\Arr_i$}
  \UnaryInfC{$\fCenter \lambda x . x
    : (\forall X. (A \Arr X) \Arr (B \Arr X) \Arr X)
    \Arr (A \Arr C) \Arr (B \Arr C) \Arr C $}
\end{scprooftree}

When we make the substitution $X := C$ in types we are actually assuming
a more general context where $C$ is a type, no problem.
