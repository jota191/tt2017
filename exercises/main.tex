\documentclass{article}
\usepackage[english]{babel}
%\usepackage[latin1]{inputenc}
%\usepackage[T1]{fontenc}
%\usepackage{amsfonts}
\usepackage[fleqn]{amsmath}
\usepackage{amsthm}
\usepackage{amssymb}
%\usepackage{verbatim}
\usepackage{enumerate}
\usepackage{graphicx}
\usepackage{caption}
\usepackage[utf8]{inputenc}
\usepackage{bussproofs}


%\newcommand{\R}{\mbox{$\mathbbm{R}$}}
\newcommand{\R}{\mathbb{R}}
\newcommand{\N}{\mathbb{N}}
% Definici\'{o}n de la hoja
\textheight 24cm \textwidth 17cm \topmargin -10mm \oddsidemargin
-3mm \evensidemargin -3mm

\pagenumbering{alph}
\setlength{\parindent}{0pt}

\renewcommand{\baselinestretch}{1.2}

\pagestyle{empty} \pagenumbering{arabic}

\newtheorem{lemma}{Lemma}
\newtheorem{theorem}{Theorem}


\title{Resolution of exercises}
\author{Juan García Garland}
\date{\today}

\begin{document}

\maketitle

\newpage
  
\section*{Some Remarks}

\subsection*{Notation}
We write $\lambda f g . t$ instead of $\lambda f . \lambda g . t$ as usual.
we write $\lambda \_ . t$ to construct a term that "eats an argument". So
$\_$ is a variable such that $\_ \notin FV(t)$



\section*{Exercise 5}

A good idea to think how to construct the terms is to reason over the
types that they should have. Even if the system is untyped, this
strategy works. Type systems usually restrict which terms can be constructed,
but they do not affect reduction. If we construct a term assuming a type
system and it reduces as we want, it will work in the untyped calculus.

In this exersise whe should encode the universal property of coproduct.
$\Sigma_i$ are the constructors (usually $inl$, $inr$ in functional
programming literature), so they have type $A_i \rightarrow A_1 + A_2$.

Since we do not have a primitive way to construct a sum type (in the way
we defined our calculus), we have to encode it in
the same way we encode, for example Booleans with the type
$ X \rightarrow X \rightarrow X$ (given two terms of type $X$,
in which one you choose you have a bool of information).

Using this reasoning, a good way to type the terms is:
$$\Sigma_i : A_i \rightarrow
            (A_1 \rightarrow B) \rightarrow
            (A_2 \rightarrow B) \rightarrow
            B
$$


So
$$
    \Sigma_1 = \lambda t f g . f t \\
$$
$$
    \Sigma_2 = \lambda t f g . g t
$$

are good definitions.

$C$ is the destructor of this case analysis, it should have type

$$
    C : A_1 + A_2 \rightarrow
       (A_1 \rightarrow B) \rightarrow
       (A_2 \rightarrow B) \rightarrow
       B
$$

So
$$
     C = \lambda c f g . c f g \\
$$

is a possible definition (recall: unfold the definition of $A_1 + A_2$
and this type checks).

Finally we can verify:

$$
    \underline{C} (\Sigma_1 t) f_1 f_2
    \succ^{3}  (\Sigma_1 t) f_1 f_2
    = (\underline{\lambda t f g} . f t) t f_1 f_2
    \succ^{3} f_1 t
$$
$$
    \underline{C} (\Sigma_2 t) f_1 f_2
    \succ^{3}  (\Sigma_2 t) f_1 f_2
    = (\underline{\lambda t f g} . g t) t f_1 f_2
    \succ^{3} f_2 t
$$

note: Actually $C = Id$ works fine and we need less reductions, I preferred
this presentation because is inmediate from some program calculation from
types, I think also that this technique will be useful to derive
harder $\lambda$-programs.

note: This is probably not the unique way to code coproducts.

note: $C$ is actually a (prefix) {\tt case} construction,
($C \: u \: f \: g$) actually models:

{\tt
  case u of\\
  {\setlength\parindent{24pt}
    \indent inl t  $\rightarrow$ f t\\
    \indent inr t  $\rightarrow$ g t
  }
}


\newpage

\section*{Exercise 7}

A great advantage of Scott numerals is that the predecessor function can be
encoded efficiently. On the other hand, functions like sum, that are computed
in constant time in Church's implementation, are linear in this case.

Let $ \bar{0} := \lambda f x . x $ and $ S := \lambda n f x . f n $

It is not hard to show that

$ \bar{n}
= \lambda f x . f (\lambda f x . f ( ...(\lambda f x . f (\lambda f x. x))))$
(each successor application is a double abstraction, applying the first
argument, where $\bar{0}$ is the second double abstraction using the second
argument).

\subsection*{Zero test}

$Zero := \lambda m t f . m (\lambda \_ . f) t$ where


$Zero$ applied to a numeral $m$ must be a boolean, so there are two extra
arguments (recall: $True = \lambda t f.t$, $False = \lambda t f . f$).

Then apply $m$ to a function that eats an argument and returns $f$, and to
$t$. If $m$ is $\bar{0}$ we use the second argument ($t$), else the result is
apply $f$ to the predecessor of $m$, eating it and returning $f$.

$$
Zero \; \bar{0} = Zero \; (\lambda f x . x) =
(\lambda m t f . m (\lambda \_ . f) t) (\lambda f x . x) \succ
\lambda t f . (\lambda f x . x) (\lambda \_ . f) t  \succ^{2} \lambda t f .t
$$

$$
Zero \; \bar{0} = Zero \; (\lambda f x . f N) =
(\lambda m t f . m (\lambda \_ . f) t) (\lambda f x . f N) \succ
\lambda t f . (\lambda f x . f N) (\lambda \_ . f) t
\succ^{2} \lambda t f . (\lambda \_ . f) N \succ \lambda t f . t
$$

Note that we can use zero test as a case analysis. EXPLAIN


\subsection*{Predecessor}

Erase the first two abstractions and the application of the first
parameter. So, apply the numeral to the identity and somewhat else.

$Pred := \lambda m . m \: (id) \: y$

Thus:

$$
Pred \; \bar{0} = (\lambda m . m \: (id) \: y) (\lambda f x. x)
\succ (\lambda f x. x) (id) y \succ^{2} y
$$

So $y$ is the value of $Pred \; \bar{0}$, we can redefine
$Pred := \lambda m . m \: (id) \: \bar{0}$ if we want that
$Pred \: \bar{0} \succ^{*} \bar{0}$, wich is a reasonable implementation.

$$
Pred \; (S N) = (\lambda m . m \: (id) \: y) (\lambda f x. f N)
\succ (\lambda f x. f N) (id) y \succ^{2} (id) N \succ N
$$

\subsection*{Recursor}

It is programmed exactly in the same way than for Church numerals, since we
do not touch the "low level implementation" (of course, all advantages of
modularization are also true in this level).

R must take a function to apply in the recursive case, and value to return
in the zero case, it must satisfy this equations:


\begin{align}
  R f a \bar{0} &\equiv_{\beta} a \\
  R f a (S N) &\equiv_{\beta} f (S N) (R f a N)
\end{align}

The second equation can be rewtited as

\begin{align}
  R f a N &\equiv_{\beta} f N (R f a (Pred N))
\end{align}

To avoid pattern matching in the numeral argument. Then, combine both (1) and
(3) as:

\begin{align}
  R \: f \: a \: N &\equiv_{\beta} Zero \: N \:a (f N (R \: f \:a (Pred\: N)))
\end{align}

Finally we use the usual trick to write recursive functions. Given the $Y$
fixpoint combinator, let $R := Y \: R'$ where

$$
R' = \lambda R \:f \:a \:n . Zero \: n\: a\: (f\: n (R \:f \:a \:(pred \: n)))
$$

\section*{Adition, Multiplication, Exponentiation}

Since the $Pred$ function reduces in constant time, these functions
can be implemented using the recursor without much overhead.

$Add := \lambda \: m \:n . R \: (\lambda \_ . S) \: m \: n$

$Mul := \lambda \: m \:n . R \: (\lambda \_ . Add \: m)\: \bar{0} \: n$

$Exp := \lambda \: m \:n . R \: (\lambda \_ . Mul \: m)\: \bar{1} \: n$

\section*{Equality test and Comparison}

First define

$Sub := \lambda \: m \:n . R \: (\lambda \_ . Pred) \: m \: n$

The substraction function (with the implementation od $Pred$ such that
$Pred \: \bar{0} = \bar{0}$).

Then:

$Leq = \lambda m \: n . Zero (Sub \: m \: n)$

$Eq = \lambda m \: n . And (Leq \: m \: n)(Leq \:  n \: m) $

where $And = \lambda a \: b \: t \: f . a \: b (\lambda t \: f . f)$

(Can be implemented more efficiently, of course)


\newpage

\section{Exercise 10 - Church-Rosser Theorem}

We must show that $\succ^{*}$ is confluent.

Define $D : \Lambda \rightarrow \Lambda$ as:
\begin{align*}
 & D(x) = x \\
 & D(\lambda x.t) = \lambda x. D(t)\\
 & D((\lambda x.t)u) = D(t)[x:=D(u)]\\
 & D(tu) = D(t) D(u) \text{\; if t is not an abstraction}
\end{align*}



\begin{lemma}
  If $t \succ^{*} t'$ and $u \succ^{*} u'$ then $t[x:=u] \succ^{*} t'[x:=u']$.
\end{lemma}
\begin{proof}
  Well known, it follows by induction over $t$ using the substitution lemma.
\end{proof}

We prove the following lemmas by induction over a term. We separate the
application in cases depending if the first term is an abstraction or not,
since the definition of $D$ is different in each case.

\begin{lemma}
 $ t \succ^{*} D(t)$
\end{lemma}

\begin{proof}
  By induction over $t$.

  \begin{itemize}
  \item
    If $t = x$ this is trivial since $D(t) = x$, so $t \succ^{0} D(t)$.
  \item
    If $t = \lambda x. a$, by induction hypothesis $a \succ^{*} D(a)$, since
    $a$ is smaller than $t$,
    so $\lambda x. a \succ^{*} \lambda x . D(a) = D(\lambda x. a)$
  \item
    If $t = (\lambda x . a) b$, then
    $(\lambda x . a) b \succ a[ x:= b] \succ^{*}_{hip} D(a)[x:=D(b)]
    = D((\lambda x.a) b)$ using Lemma 1 and induction hypothesis.
  \item
    Finally if $t = a b$ where $a$ is not an abstraction,
    \begin{flalign*}
      & a b &\succ^{*} &\text{ \{ by induction hypothesis, twice \} }
      \hskip0.7\textwidth\\
      & D(a)D(b) &=  &\text{ \{ by definition of $D$ \}}\\
      &D(a b)
    \end{flalign*}
  \end{itemize}
\end{proof}


\begin{lemma}
 If $ t \succ t'$ then $t' \succ^{*} D(t)$
\end{lemma}

\begin{proof}
  By induction over $t$ (considering the rules for beta reductions).
  \begin{itemize}
  \item
    If $t = x$ trivial (actually not possible, since $t$ has no redex).
  \item
    If $t= \lambda x . a$, then necessarly $t' = \lambda x. a'$
    where $a \succ a'$ (there is only one redex rule for abstraction,
    we reason by inversion). Then:
    \begin{flalign*}
      & t' &= & \\
      & \lambda x . a' &\succ^{*} &\text{\{ by induction hypothesis in $a$\}}
      \hskip0.7\textwidth\\
      & \lambda x . D(a) &=  &\text{\{ by definition of $D$\}} \\
      & D(\lambda x.a ) = D(t)
    \end{flalign*}
    \item
      If $t = (\lambda x . a) b$ it can reduce in one step to
      $(\lambda x . a') b$,
      $(\lambda x . a) b'$ or $a[x:= b]$ (where $a \succ a'$, $b \succ b'$).
      Applying one more reduction
      to the first two cases (in the outhermost redex), it is enough to show
      that $a'[x:= b]$, $a[x:= b']$, $a[x:= b]$ all reduces (in $*$ steps) to
      $D(t) = D(a) [x:= D(b)]$.
      This is trivial applying Lemma 1, and Lemma 2 and induction hypothesis
      to the instances of $t$ and $u$
      (induction to the primes, lemma to the untouched substerms).
      For instance $a[x:=b'] \succ^{*} D(a)[x:=D(b)]$ follows from applying
      Lemma 2 to a, inductive hypothesis to b', and Lemma 1 outside.
    \item
      If $t = a b$ with $a$ not being an abstraction, $t$ reduces to either
      $a' b$ or $a b'$ (where $a \succ a'$, $b \succ b'$). $D(t) = D(a)D(b)$,
      Apply Lemma 2 and induction hypothesis to the suitable subterm again,
      and we are done.
  \end{itemize}
\end{proof}

\begin{lemma}
  It $t \succ u$ then $D(t) \succ^{*} D(u)$
\end{lemma}
\begin{proof}
  Again, by induction over $t$, considering the possible beta reductions.

  \begin{itemize}
  \item
    $t=x$ is trivial, no reduction.
  \item
    $t= \lambda x. a$ can only reduce to $u = \lambda x. a'$
    with $a \succ a'$,
    then $D(t) = \lambda x. D(a)$ and $D(u) = \lambda x . D(a')$, from
    induction hypothesis ($a$ is smaller than $t$) $D(a) \succ^{*} D(a')$ and
    the result follows.
  \item
    If $t = a b$ where $a$ is not an abstraction, it is inmediate again
    applying the induction hypothesis in the term that is reduced
    (this is allways the easy -inductive- case).
  \item
    If $t = (\lambda x. a) b$, $t$ can reduce to $(\lambda x . a') b$,
    $(\lambda x . a) b'$ or $a[x:= b]$ (where $a \succ a'$, $b \succ b'$).
    In the last two cases it is easy to apply the induction hypothesis,
    in the former one, when $t$ reduces to $ u = a[x:= b]$, so we must
    show that $D(t) = D((\lambda x . a) b) = D(a)[x := D(b)]$ reduces
    (in $*$ steps) to
    $D(a[x:= b])$. We demonstrate this in the next lemma.
  \end{itemize}
\end{proof}

\begin{lemma}
  $D(t)[x:= D(u)] \succ^{*} D(t[x:=u])$
\end{lemma}

\begin{proof}
  By induction over $t$.
  \begin{itemize}
  \item
    If $t = y$, again it is trivial. If y = x both sides are $D(u)$, if
    $y \neq x$ both sides are $D(y)$.
  \item
    if $t= \lambda y . a$ then
    \begin{flalign*}
      &D(t)[x:=D(u)]  &= &\\
      &D(\lambda y.a)[x:=D(u)] &= &\text{ \{ by definition of $D$\} }
      \hskip0.7\textwidth\\
      &(\lambda y.D(a))[x:=D(u)]&= &\text{\{by definition of substitution,
        assuming $y \neq x$, otherwise rename y \}}\\
      &\lambda y.(D(a)[x:=D(u)])&\succ^{*}
      &\text{\{by inductive hypothesis\}}\\
      &\lambda y.D(a[x:=u])&= &\text{\{by definition of $D$\}}\\
      &D(\lambda y.(a[x:=u]))&= &\text{\{by definition of substitution\}}\\
      &D((\lambda y.a)[x:=u])&= &\\
      &D(t[x:=u])& &\\
    \end{flalign*}
  \item
    The application case is tricky. We must study separate cases, one more
    than before, We will see why.
    \begin{itemize}
    \item
      If $t = (\lambda y . a) b$:
      \begin{flalign*}
      &D(t)[x:=D(u)]  &= &\\
      &D((\lambda y.a)b)[x:=D(u)] &= &
      \hskip0.7\textwidth\\
      &D((\lambda y.a)b)[x:=D(u)] &= &\text{ \{ by definition of $D$\} }\\
      &(D(a)[y := D(b)])[x:=D(u)] &\succ^{*} &
      \text{ \{ by substitution lemma\} }\\
      &(D(a)[x := D(u)])[y := D(b)[x:=D(u)]] &\succ^{*}
      &\text{ \{ by inductive hypothesis \} }\\
      &(D(a[x := u)])[y := D(b)[x:=D(u)]] &\succ^{*}
      &\text{ by inductive hypothesis}\\
      &(D(a[x := u)])[y := D(b[x:=u]] &=
      &\text{ by definition of $D$}\\
      &D((\lambda y. a[x:= u])(b[x:=u])) &=
      &\text{ by definition of substitution}\\
      &D(((\lambda y. a)[x:= u])(b[x:=u])) &=
      &\text{ by definition of substitution}\\
      &D((\lambda y. a)b[x:= u]) &=
      &\\
      &D(t[x:= u]) &
      &
      \end{flalign*}
    \item
      If $t = a b$, and a is not a variable, then:
      \begin{flalign*}
        &D(t)[x:=D(u)]  &= &\\
        &D(a b)[x:=D(u)]&= &\text{ \{ by definition of $D$\} }
        \hskip0.7\textwidth\\
        &(D(a) D(b))[x:=D(u)]&= &\text{ \{ by definition of substitution\}}\\
        &(D(a)[x:=D(u)]) (D(b)[x:=D(u)])&= &\text{\{by hypothesis, twice\}}\\
        &D(a[x:=u]) D(b[x:=u])&= &\text{ \{ by definition of $D$\} }\\
        &D(a[x:=u] b[x:=u])&= &\text{ \{ by definition of substitution\} }\\
        &D((a b)[x:=u])&= &\\
        &D(t[x:=u])& &
      \end{flalign*}
      \item
        If $t = x b$
        this case is different to the previous one since the substitution
        could generate a new redex. Note that if $t = y b$
        where $y \neq x$ the proof is like in the former case.
      \begin{flalign*}
        &D(t)[x:=D(u)]  &= &\\
        &D(x b)[x:=D(u)]&= &\text{ \{ by definition of $D$\} }
        \hskip0.7\textwidth\\
        &(D(x) D(b))[x:=D(u)]&= &\text{ \{ by definition of substitution\}}\\
        &(D(x)[x:=D(u)]) (D(b)[x:=D(u)])&=
        &\text{ \{ by definition of substitution and $D$ \} }\\
        &(D(u)) (D(b)[x:=D(u)])&=
        &\text{ \{ applying inductive hypothesis\} }\\
        &(D(u)) (D(b[x:=u]))&=
        &\text{ \{ applying substitution backwards\} }
      \end{flalign*}
      If $u$ is not an abstraction there is no problem (rewrite
      the definition of $D$ backwards and so on). If $u = \lambda y.c$ then:
      \begin{flalign*}
        &(D(u)) (D(b[x:=u]))&=
        &
        \hskip0.7\textwidth\\
        &(D(\lambda y.c)) (D(b[x:=u]))&=
        &\text{ \{ by definition of $D$\} }\\
        &(\lambda y.D(c)) (D(b[x:=u]))&\succ
        &\text{ }\\
        &D(c)[y:= (D(b[x:=u]))]&=
        &\text{ \{ rewriting $D$\}}\\
        &D((\lambda y.c)(b[x:=u])) &= &\\
          &D(u (b[x:=u]))&= & \text{ \{ by definition of substitution\}}\\
            &D((x [x:= u]) (b[x:=u]))&= &
              \text{ \{ by definition of substitution\}}\\
              &D(x u [x:=u])& &
      \end{flalign*}
    \end{itemize}
  \end{itemize}
\end{proof}

\begin{lemma}
  $t \succ^{*} u$ implies
  $D(t) \succ^{*} D(u)$
\end{lemma}
\begin{proof}
  By induction over the number of steps of the $t \succ^{*} u$ reduction.
  If $n = 1$ then this is an instance of Lemma 4.

  If $t\succ^{n} u$ then consider the first reduction, so $t \succ t'$,
  $t' \succ^{n-1} u$. By induction hypothesis $D(t') \succ^{*} D(u)$,
  and by Lemma 4 $D(t) \succ^{*} D(t')$. By transitivity of $\succ^{*}$
  follows that $D(t) \succ^{*} D(u)$.
\end{proof}


\begin{lemma}
  $t \succ^{n} u$ implies
  $u \succ^{*} D^{n}(t)$
\end{lemma}
\begin{proof}
  Again, by induction over the length of reduction.

  For the base case, this is exactly Lemma 3.

  For $t \succ^{n} u$ with $n>1$ consider again the first reduction,
  so $t \succ t'$ and $t' \succ^{n-1} u$. $t' \succ^{*} D(t)$ by Lemma 3, and
  by induction hypothesis $u \succ^{*} D^{n-1}(t')$, Applying Lemma 6
  $D^{n-1}(t') \succ^{*} D^{n-1}(D(t)) = D^{n}(t)$. By transitivity of
  $\succ^{*}$ then $u \succ^{*} D^{n}(t)$ as we claim.
\end{proof}

\begin{theorem}[Church-Rosser Theorem]
  The relation $\succ^{*}$ is confluent. This is, if $t \succ^{*} t_1$
  and $t \succ^{*} t_2$ then there is a term $v$ such that $t_1 \succ^{*} v$
  and $t_2 \succ^{*} v$.
\end{theorem}

\begin{proof}
  If $t \succ^{n} t_1$ and $t \succ^{m} t_2$, then applying Lemma 7
  $t_1 \succ^{*} D^n(t)$ and $t_2 \succ^{*} D^m(t)$.
  Without loss of generality assume that $m \geq n$. It is easy to show that
  $D^n(t) \succ^{*} D^m(t)$ (Applying Lemma 2 $m-n$ times,
  $t \succ^{*} D^{m-n}(t)$, then Apply Lemma 6 $n$ times, or more formal, we
  can prove by an easy induction).

  Then by transitivity of $\succ^{*}$ is true that $t_1 \succ^{*} D^m(t)$,
  take $u = D^m(t)$.
\end{proof}


\newpage

\section{Exercise 21}

In general, in simply typed $\lambda$-calculus
if we can derive a typing judgement $\Gamma \vdash t : A$
then we can derive it extending the context, since when checking the context
the only interesting thing is if typing judgements belong to it
or not.

Also, if we can make a derivation in HOL,
expanding the context (of assumptions) we can also make the derivation,
just ignore the new axioms. (Weakening).

Let's be more precise:

\begin{lemma}
  
  If $\Gamma \vdash t : A$ and $x \notin Dom(\Gamma)$
  then $\Gamma, x:X \vdash t : A$
\end{lemma}
\begin{proof}
  This is straightforward by induction on $t$. Since the typing derivation
  is syntax directed, for each shape of $t$ there is only one rule
  that could be applied in that case.
  \begin{itemize}
  \item
    If $t = x$ then the only possible derivation is
    \begin{prooftree}
      \alwaysNoLine
      \AxiomC{$\vdots$}
      \UnaryInfC{ $\Gamma(x) = A$ }
      \alwaysSingleLine
      \UnaryInfC{$x : A$}
    \end{prooftree}

    and then the following derivation is well-formed:
    
    \begin{prooftree}
      \alwaysNoLine
      \AxiomC{$\vdots$}
      \UnaryInfC{ $ (\Gamma,y : T) (x) = A$ }
      \alwaysSingleLine
      \UnaryInfC{ $x : A$ }
    \end{prooftree}

    since the extension of a function does not change the value in
    a point of the previous domain.
    
  \item
    If $t = \lambda x.b$ then the typing derivation must be like:
    
    \begin{prooftree}
      \alwaysNoLine
      \AxiomC{\vdots}
      \UnaryInfC{ $\Gamma, x:A \vdash b : B $ }
      \alwaysSingleLine
      \UnaryInfC{$\Gamma \vdash  \lambda x. b : A \Rightarrow B$}
    \end{prooftree}

    Then the following derivation is well formed:
    \begin{prooftree}
      \alwaysNoLine
      \AxiomC{\vdots}
      \UnaryInfC{ $\Gamma, x:A, y:T \vdash b : B $ }
      \alwaysSingleLine
      \UnaryInfC{$\Gamma, y:T \vdash  \lambda x. b : A \Rightarrow B$}
    \end{prooftree}

    applying the induction
    hypothesis to the term $b$.
  \item
    The application case is even more
    trivial since there is no manipulation of the context.
    
    If $t = u v$ then the root of the typing proof is:

    \begin{prooftree}
      \alwaysNoLine
      \AxiomC{\vdots}
      \UnaryInfC{ $\Gamma \vdash a : A\Rightarrow B$ }
      \AxiomC{\vdots}
      \UnaryInfC{ $\Gamma \vdash b : A $ }
      \alwaysSingleLine
      \BinaryInfC{$\Gamma \vdash  a \, b : B$}
    \end{prooftree}
\newpage
    Then the following derivation is well formed: 
    
    
    \begin{prooftree}
      \alwaysNoLine
      \AxiomC{\vdots}
      \UnaryInfC{ $\Gamma, x : T \vdash a : A \Rightarrow B$ }
      \AxiomC{\vdots}
      \UnaryInfC{ $\Gamma, x : T \vdash b : A $ }
      \alwaysSingleLine
      \BinaryInfC{$\Gamma \vdash  a \, b : B$}
    \end{prooftree}

    applying induction hypothesis to $a$ and $b$.
  \end{itemize}
\end{proof}


\begin{lemma} [Weakening in simply typed $\lambda$-calculus]
  
  If $\Gamma \vdash t : A$ and $\Delta$ is a context with disjoint domain than
  $\Gamma$
  then $\Gamma, \Delta \vdash t : A$
\end{lemma}
\begin{proof}
  Trivial by induction on the length of $\Delta$, applying the previous lemma.
  Note that $\Gamma$ is finite, so we can do this.
\end{proof}


\begin{lemma} [Weakening in HOL]
\label{weakening}
  
  If $\gamma ; \Gamma \vdash t : A$ is derivable
  then $ \gamma ; \Gamma,\Delta \vdash t : A$ is derivable, assuming
  $\Gamma$ and $\Delta$ with disjoint domains.  
\end{lemma}

\begin{proof}
  I'll be more sloppy here since otherwise this is very verbose.

  This lemma is trivial by induction in the size of $\Delta$ once we prove
  that we can add a proposition to a context. (Note again that we
  are assuming that contexts are finite).

  To prove this, we can use induction over the rules. For example for
  the Axiom rule
  we only check that $A \in \Gamma$, so this is true when we extend
  $\Gamma$.

  For inductive rules in every one either we do not touch
  the context or we check if something belongs to it, which is conservative
  when we extend it.
\end{proof}

\begin{theorem}[Exercise 21]
  If $\: \gamma ; \Gamma \vdash A$ is derivable then $\: \gamma, \delta ;
  \Gamma, \Delta \vdash A$ is derivable.
\end{theorem}

If $\: \gamma \vdash A : o \:$ is derivable then we have
$\:\gamma,\delta \vdash A : o \:$
derivable by weakening in $\lambda$-calculus.
So if $\Gamma$ is a well formed
set of axioms under the context $\gamma$ of the underlying simply typed
lambda calculus, it is well formed under $\gamma, \delta$.

So $\gamma; \Gamma \vdash A$ derivable implies
$\gamma,\delta; \Gamma \vdash A$ derivable.
Applying weakening for HOL then $\gamma,\delta; \Gamma, \Delta \vdash A$ can
be derived, as we wanted.

This rule is not derivable. We are not doing sequent calculus so there
is no way to introduce formulas in the left. Formally this is proved
trivially by induction over the rules.
Note that when applying a rule, the
context is decreasing, by induction this is true for a proof,
which contradicts the possibility of derive this rule (At least if $\delta$
and $\Delta$ are nonempty).



\newpage

\section{Exercise 24}

Let's start for what we can prove.
The second rule is ok even without weakening.

\begin{theorem}

  The following rule is derivable, assuming $x \notin FV(B)$:
  
  
  \begin{prooftree}
    \AxiomC{$\gamma ; \Gamma \vdash \exists_s x . A$}
    \AxiomC{$\gamma, x:s ; \Gamma \vdash A \Rightarrow B$}
       \RightLabel{$\exists_e$}
    \BinaryInfC{$\gamma; \Gamma \vdash B$}
  \end{prooftree}

\end{theorem}


% some definitions

\newcommand{\existsX}{\exists_s x}
\newcommand{\existsXUnfolded}
           {\lambda X. \forall_o Y .
             (\forall_s x. X x \Rightarrow Y) \Rightarrow Y }

\newcommand{\existsApplied}
           {\forall_o Y .
             (\forall_s x. A x \Rightarrow Y)
             \Rightarrow Y }

           
\begin{proof}

  Note that $\existsX A \equiv_{\text{def}}
  (\existsXUnfolded) A \succ_{\beta} \existsApplied $

\begin{prooftree}

  \AxiomC{$\gamma, x:s ; \Gamma \vdash A \Rightarrow B$}
     \RightLabel{$\forall_i$}
  \UnaryInfC{$\gamma ; \Gamma \vdash \forall x_s . (A x \Rightarrow B)$}
  
  \AxiomC{$\gamma \vdash B : o$}
  
  \AxiomC{$\gamma \vdash \existsX A : o $}
  \AxiomC{$  \existsApplied \equiv_{\beta}  \existsX A$}
  \AxiomC{$\gamma; \Gamma \vdash \existsX A$}
     \RightLabel{$\beta$}
  \TrinaryInfC{$\gamma; \Gamma \vdash \existsApplied $}

     \RightLabel{$\forall_e$}
  \BinaryInfC{$ \gamma ; \Gamma \vdash
    (\forall_s x. A x \Rightarrow B) \Rightarrow B$}

     \RightLabel{$\Rightarrow_e$}
  \BinaryInfC{$\gamma ; \Gamma \vdash B$}
\end{prooftree}

Note that leaves are axioms, a $\beta$-equivalence,
and $\gamma \vdash B:o$ which must be true since
$ \gamma, x:s \vdash (A \Rightarrow B) : o$


\end{proof}


To derive a left rule, we need to manipulate context:

\begin{theorem}[Left $\exists$-elimination]
  
  The following rule is admissible:
  

  \begin{prooftree}
    \AxiomC{$\gamma;\Gamma, \exists_s x . A \vdash B$}
    \RightLabel{$L\exists_{e}$}
    \UnaryInfC{$\gamma;\Gamma, A[x:=e] \vdash B$}
  \end{prooftree}

\end{theorem}

\begin{proof}

  We write a derivation, note that leaves are premises. Also note that we use
  weakening and a rule that was not derivable without it ($\exists_i$). So
  we are proving admissibility:

  \begin{prooftree}
    
    \AxiomC{$ \gamma;\Gamma, \exists_s x . A \vdash B$}
    \RightLabel{$\forall_i$}
    \UnaryInfC{$\gamma; \Gamma \vdash \exists_s x . A \Rightarrow B $}
    \RightLabel{$wk$}
    \UnaryInfC{$\gamma; \Gamma, A[x:=e] \vdash \exists_s x . A \Rightarrow B $}
    \AxiomC{$\gamma \vdash e : s$}
    \AxiomC{}
    \RightLabel{Ax}
    \UnaryInfC{$\gamma, \Gamma, A[x:=e]\vdash A [x:=e] $}
    \RightLabel{$\exists_i$}
    \BinaryInfC{$\gamma;\Gamma, A[x:=e] \vdash \exists_s x . A$}

    \RightLabel{$\Rightarrow_e$}
    \BinaryInfC{$\gamma;\Gamma, A[x:=e] \vdash B$}
  \end{prooftree}

\end{proof}

The first rule given is also admissible. Again, now that we showed some formal
proofs, let's be more sloppy here.
(And also because actually we will be using
admissible but not derivable rules everywhere, to be really formal we need
some boring work before).

We want to prove $\gamma; \Gamma \vdash \exists_s x . A$, if we unfold the
definition of $\exists$ and do a beta reduction on the goal this is
$\gamma; \Gamma \vdash
\forall Y. (\forall_s x. A(x) \Rightarrow Y) \Rightarrow Y $.
Introducing the universal quantifier and the first paremeter of the arrow,
the goal is
$\gamma; \Gamma, Y:o, \forall_s x. A(x) \Rightarrow Y \vdash Y $.
Since we can derive $A[x:=e]$, applying
$\forall_s x. A(x) \Rightarrow Y$ to $e$ and $A[x:=e]$ we have what we want.

Note that we are using a lot of non-primitive rules here. 



\newpage


\section{Exercise 26}

Again we will not show a strictly formal proof, but they can be implemented
trivially. We assume implicitly that we previously derived some usual rules
such as $RAA$ or $\iff_{intro}$. So to prove equivalence we prove
double implication. Also in the second equivalence we make a proof
by contradiction, and use $\neg \exists_s x. \neg A \iff \forall_s x . A $.

\begin{theorem}
  $A[x:=\epsilon_s (\lambda x. A)] \Leftrightarrow \exists_s x . A$
\end{theorem}
\begin{proof}

  The left to right implication is trivial applying the rule $\exists_i$
  of the exercise 24. We only need that
  $\gamma \vdash \epsilon_s (\lambda x. A) : s$ which is well typed.

  For the opposite we apply the $\exists_e$ rule of the exercise 24.

\end{proof}


\begin{theorem}
  $A[x:=\epsilon_s (\lambda x. \neg A)] \Leftrightarrow \forall_s x . A$
\end{theorem}
\begin{proof}
  We are using classical logic so assume by contradiction
  $\exists_s x . \neg A$. Then applying $\exists_e$ with
  $\neg A \Rightarrow \neg A [x:=\epsilon_s (\lambda x. \neg A)]$
  (which is provable) we prove $\neg A [x:=\epsilon_s (\lambda x. \neg A)]$.
  By contraposition, we proved
  $A[x:=\epsilon_s (\lambda x. \neg A)] \Rightarrow \forall_s x . A$ (using
  De Morgan's law and double negation elimination).

  The opposite is trivial, apply the $\epsilon$ rule.
\end{proof}


\newpage

\section{Exercise 30}

This was solved in Coq ({\tt Arith.v}).

\section{Exercise 31}

Also implemented in {\tt Arith.v}.


\newpage

\newcommand{\NatF}
           {\forall X. (X \Rightarrow X) \Rightarrow X \Rightarrow X}

%\newcommand{\N}
%           {\mathfrak{N}}
           
\newcommand{\Arr}
           {\Rightarrow}

           
\section{Exercise 37 - Arithmetic functions in SystemF}

One way to solve this exercise is to do in the same way then in Exercise
14, constructing a proof tree and collecting equations
to form a system. Then find an unifier.

This could work by hand in this small examples, as a guide to figure out
types, but not in general as SystemF typing is not decidable in general
[Wells96]. %TODO bibtex blah blah
(Proofs are no longer syntax directed and unification is no longer first
order).

Since we have to write {\bf a} type, and it is very intuitive which one
we want, We'll show a typing derivation for that type instead of trying
to infer one from scratch.



We will use the notation
$\N := \NatF$, folding and unfolding this definition when we
need it.

\subsection*{Addition}


\begin{theorem}
  Let $Add = \lambda m n f x . m f (n f x)$, then:
  $\emptyset \vdash Add : \N \Arr \N \Arr \N$
\end{theorem}

\newenvironment{scprooftree}[1]%
  {\gdef\scalefactor{#1}\begin{center}\proofSkipAmount \leavevmode}%
  {\scalebox{\scalefactor}{\DisplayProof}\proofSkipAmount \end{center} }


\begin{proof}
  We construct a proof tree:

  
  \begin{scprooftree}{0.80}
    \def\fCenter{\vdash}

    \AxiomC{}
    \RightLabel{Ax}
    \UnaryInfC {$\Gamma \fCenter\ m : \NatF$}
    \RightLabel{$\forall_e$}
    \UnaryInfC{$\Gamma 
      \fCenter\ m : (X\Arr X)\Arr X \Arr X$}

    \AxiomC{}
    \RightLabel{Ax}
    \UnaryInfC{$\Gamma \fCenter f : X\Arr X$}
    \RightLabel{$\Arr_e$}
    \BinaryInfC{$ \Gamma \fCenter\ m f : X \Arr X $}
    

    %----
    \AxiomC{}
    \RightLabel{Ax}
    \UnaryInfC{$\Gamma \fCenter n : \forall X. (X \Arr X)\Arr X \Arr X$}
    \RightLabel{$\forall_e$}
    \UnaryInfC{$\Gamma \fCenter n : (X \Arr X)\Arr X \Arr X$}
    \AxiomC{}
    \RightLabel{Ax}
    \UnaryInfC{$\Gamma \fCenter\ f : X\Arr X$}
    \RightLabel{$\Arr_e$}
    \BinaryInfC{$\Gamma \fCenter\ n f : X\Arr X$}

    \AxiomC{}
    \RightLabel{Ax}
    \UnaryInfC{$\Gamma \fCenter\ x : X$}
    \RightLabel{$\Arr_e$}
    \BinaryInfC{$ \Gamma \fCenter\ n f x : X $}
                                              \RightLabel{$\Arr_e$}
    \BinaryInfC{$ (\Gamma \equiv) m,n:\N, f: X\Rightarrow X, x:X
                 \fCenter\ m f (n f x) : X $}
                                              \RightLabel{$\Rightarrow_i$}
    
    \UnaryInfC{$ m,n:\N, f: X\Rightarrow X \fCenter\ \lambda x. m f (n f x)
      : \Rightarrow X \Rightarrow X $}
                                              \RightLabel{$\Rightarrow_i$}
    
    \UnaryInfC{$ m,n:\N \fCenter\ \lambda f x. m f (n f x)
      : (X\Rightarrow X) \Rightarrow X \Rightarrow X $}
                                              \RightLabel{$\forall_i$}
    
    \UnaryInfC{$ m,n:\N \fCenter\ \lambda f x. m f (n f x) :\NatF(\equiv\N)$}
                                              \RightLabel{$\Rightarrow_i$}
    \UnaryInfC{$ m:\N \fCenter\ \lambda n f x. m f (n f x) : \N\Arr\N$}
                                              \RightLabel{$\Rightarrow_i$}
    \UnaryInfC{$\vdash \lambda m\ n\ f\ x. m f (n f x) : \N \Arr \N \Arr \N$}
    
  \end{scprooftree}
\end{proof}



\subsection*{Multiplication}

\begin{theorem}
  Let $Mult = \lambda m n f x . m (n f) x$, then:
  $\emptyset \vdash Mult : \N \Arr \N \Arr \N$
\end{theorem}
\begin{proof}
  
  Again, we construct the derivation. To have a smaller proof we apply
  multiple elimination rules in one step (the usual intros tactic).
  
  \begin{scprooftree}{1.0}
    \def\fCenter{\vdash}
    
    \AxiomC{$(\Gamma \equiv) m,n:\N,f:\N\Arr\N,x:\N \fCenter m(n f) x :X$}
    \RightLabel{$\Arr_i,\Arr_i,\forall_i,\Arr_i,\Arr_i$}
    \UnaryInfC{$\fCenter \lambda m n f x . m (n f) x : \N\Arr\N\Arr\N$}
  \end{scprooftree}
\end{proof}


\newpage

\section{Exercise 38 - Product and Coproduct in System F}

%\newcommand{\Arr}{\Rightarrow}

Terms are the same as in simply typed lambda calculus.
Again we will ommit computing reductions (obviowsly we must prove that
this terms reduces to what we want) since this was
well studied in class and at IMERL's hall, in previous exercises,
(and actually it is very easy in most cases).

Type derivations, in the other hand, will be provided.


\subsection*{Product}

Let:

\begin{align*}
  A \times B := \forall X. (A\Arr B\Arr X) \Arr X \nonumber
\end{align*}
  
  The type of the -unique- constructor $pair$, and destructors $\pi_1$ and
  $\pi_2$ are:

\begin{align}
  pair  :& A \Arr B \Arr A\times B  \nonumber \\
  \pi_1 :& A \times B \Arr A        \nonumber \\
  \pi_2 :& A \times B \Arr B        \nonumber
\end{align}
  
  where they are defined as:

\begin{align}
  pair :=& \lambda a b p . p a b        \nonumber \\
  \pi_1:=& \lambda p. p (\lambda t f.t) \nonumber \\
  \pi_2:=& \lambda p. p (\lambda t f.f) \nonumber
\end{align}

We do not use actual booleans since they would not be well typed here.

Denoting $pair\ a\ b$ as $(a,b)$ it is easy to show that:

\begin{align*}
  \pi_1 (a,b) &\succ^{*} a \\
  \pi_2 (a,b) &\succ^{*} b
\end{align*}

The typing proofs are the following:

For $\pi_1$ (for $\pi_2$ is analogous):

\newcommand{\AxB}{ \forall X. (A\Arr B \Arr X) \Arr X }

\begin{scprooftree}{0.95}
  \def\fCenter{\vdash}

  \AxiomC{}
  \RightLabel{Ax}
  \UnaryInfC{$p : \AxB \fCenter p : \forall X. (A\Arr B \Arr X) \Arr X$}
  \RightLabel{$\forall_e$ $[X/A]$}
  \UnaryInfC{$p : \AxB \fCenter p : (A\Arr B \Arr A) \Arr A$}

  \AxiomC{}
  \RightLabel{Ax}
  \UnaryInfC{$p : \AxB, t:A, f:B \fCenter t :A $}
  \RightLabel{$\Arr_i, \Arr_i$}
  \UnaryInfC{$p : \AxB \fCenter  (\lambda t f. t) : (A\Arr B \Arr A)$}
  \BinaryInfC{$ p : \forall X. (A\Arr B \Arr X) \Arr X
    \fCenter p (\lambda t f. t) : A$}
  \RightLabel{$\Arr_i$}
  \UnaryInfC{$\fCenter \lambda p.p (\lambda t f. t) : A\times B\Arr A$}
\end{scprooftree}


\newpage
For $pair$:

\begin{scprooftree}{0.95}
  \def\fCenter{\vdash}

  \AxiomC{}
  \RightLabel{Ax}
  \UnaryInfC{$\Gamma \fCenter p : A \Arr B \Arr X$}
  
  \AxiomC{}
  \RightLabel{Ax}
  \UnaryInfC{$\Gamma \fCenter a : A$}
  \RightLabel{$\Arr_e$}
  \BinaryInfC{$\Gamma \fCenter p \, a : B \Arr X$}
  
  \AxiomC{}
  \RightLabel{Ax}
  \UnaryInfC{$ \Gamma \fCenter b : B$}
  \RightLabel{$\Arr_e$}
  \BinaryInfC{$(\Gamma := )a : A, b:B, p : A \Arr B \Arr X \fCenter p a b : X$}
  \RightLabel{$\Arr_i, \Arr_i, \forall_i, \Arr_i$}
  \UnaryInfC{$ \fCenter \lambda a b p . p a b : A \Arr B \Arr A \times B $}
\end{scprooftree}


\subsection*{Co-Product}

Let:

\begin{align}
  A + B := \forall X. (A\Arr X) \Arr (B\Arr X) \Arr X \nonumber
\end{align}

Now we have two constructors, $inl$, $inr$
and the case analysis destructor $C$:

\begin{align*}
  inl :& A  \Arr A + B   \\
  inr :& B  \Arr A + B  \\
  C   :& A + B \Arr (A \Arr C) \Arr (B \Arr C) \Arr C
\end{align*}

where they are defined as:

\begin{align*}
  inl :=& \lambda a l r . l a        \\
  inr :=& \lambda b l r . r b        \\
  C   :=& \lambda x . x
\end{align*}

Note that we chosed the simple implementation of C now,
in contrast to Exo 5, (but of course $C = \lambda c f g . c f g$ typechecks
also, they are $\eta$-equivalent).

We provide typing proofs for $inl$ and for $C$ ($inr$ is analogous to $inl$)

\begin{scprooftree}{0.95}
  \def\fCenter{\vdash}

  \AxiomC{}
  \RightLabel{Ax}
  \UnaryInfC{$ \Gamma \fCenter l : A \Arr X$}

  \AxiomC{}
  \RightLabel{Ax}
  \UnaryInfC{$ \Gamma \fCenter a : A$}
  \RightLabel{$\Arr_e$}
  \BinaryInfC{$ (\Gamma :=) a : A, l : A \Arr X, r : B \Arr X \fCenter l a : X$}
  \RightLabel{$\Arr_i, \forall_i, \Arr_i, \Arr_i$}
  \UnaryInfC{$ \fCenter \lambda a l r . l a
    : A \Arr \forall X. (A \Arr X) \Arr (B \Arr X) \Arr X $}
\end{scprooftree}


\begin{scprooftree}{0.95}
  \def\fCenter{\vdash}

  \AxiomC{}
  \RightLabel{Ax}
  \UnaryInfC{$   x : (\forall X. (A \Arr X) \Arr (B \Arr X) \Arr X
    \fCenter x: (\forall X. (A \Arr X) \Arr (B \Arr X) \Arr X$}
  \RightLabel{$\forall_e$ $[X/C]$}
  \UnaryInfC{$ x : (\forall X. (A \Arr X) \Arr (B \Arr X) \Arr X)
    \fCenter x : (A \Arr C) \Arr (B \Arr C) \Arr C$}
  \RightLabel{$\Arr_i$}
  \UnaryInfC{$\fCenter \lambda x . x
    : (\forall X. (A \Arr X) \Arr (B \Arr X) \Arr X)
    \Arr (A \Arr C) \Arr (B \Arr C) \Arr C $}
\end{scprooftree}

When we make the substitution $X := C$ in types we are actually assuming
a more general context where $C$ is a type, no problem.


\newpage

\newcommand{\hole}{ \AxiomC{$ \qed $} }



\section{Exercise 39 - Lists in System F}
 
For this excercise we could reason in the algebraic way, since we have Product
and coproduct, or directly encode lists usind the type of its constructors.
The result is the same.

Let:

\begin{align*}
  A^{*} := \forall X. X \Arr (A\Arr X\Arr X) \Arr X
\end{align*}

We want to program $nil$ and $cons$, constructors with types:

\begin{align*}
  nil  :& A^{*} \\
  cons :& A \Arr A^{*} \Arr A^{*}
\end{align*}

The reasoning when programming is the same as ever when we encode datatypes,
each (sub)term with type $A^{*}$ should have a lambda for each existing
constructor, and the application of the suitable one in the body,
to the actual parameters that the constructor has, which are furthermore closed
by outermost abstractions.

So:

\begin{align*}
  nil  =& \lambda n c . n\\
  cons =& \lambda a l n c . c a (l n c)
\end{align*}


Typing proofs:

\begin{scprooftree}{0.95}
  \def\fCenter{\vdash}

  \AxiomC{}
  \RightLabel{Ax}
  \UnaryInfC{$ n: X, c: A \Arr X \Arr X \fCenter n:X$}
  \RightLabel{$\forall_i,\Arr_i,\Arr_i$}
  \UnaryInfC{$\fCenter \lambda n c . n
    : \forall X. X \Arr (A\Arr X\Arr X) \Arr X$}
\end{scprooftree}



\begin{scprooftree}{0.86}
  \def\fCenter{\vdash}

  

  \AxiomC{}
  \RightLabel{Ax}
  \UnaryInfC{$\Gamma \fCenter c : A\Arr X\Arr X$}
  
  \AxiomC{}
  \RightLabel{Ax}
  \UnaryInfC{$\Gamma \fCenter a : A$}
  \RightLabel{$\Arr_e$}
  \BinaryInfC{$\Gamma \fCenter c a : X\Arr X$}



  \AxiomC{}\RightLabel{Ax}
  \UnaryInfC{$\Gamma \fCenter
  l : \forall X . X \Arr (A \Arr X \Arr X) \Arr X) $}

  \RightLabel{$\forall_e$}
  \UnaryInfC{$\Gamma \fCenter l : X\Arr (A\Arr X\Arr X)\Arr X$}

  \AxiomC{} \RightLabel{Ax}
  \UnaryInfC{$\Gamma \fCenter n : X$}
  \RightLabel{$\Arr_e$}
  \BinaryInfC{$\Gamma \fCenter l n : (A\Arr X\Arr X)\Arr X)$}
  \AxiomC{}\RightLabel{Ax}
  \UnaryInfC{$\Gamma \fCenter c : A\Arr X\Arr X$}
  \BinaryInfC{$\Gamma \fCenter l n c : X$}

  \RightLabel{$\Arr_e$}
  \BinaryInfC{$(\Gamma :=) a : A, l : A^{*}, n:X, c: A\Arr X\Arr X
    \fCenter c a (l n c):X$}
  \RightLabel{$\Arr_i, \Arr_i, \forall_i, \Arr_i$}
  \UnaryInfC{$\fCenter \lambda a l n c . c a (l n c) : A\Arr A^{*}\Arr A^{*} $}
\end{scprooftree}



Finally, the map function must have type:

\begin{align*}
  map :& (A\Arr B) \Arr A^{*} \Arr B^{*}
\end{align*}

(We could close type variables to have one polymorphic function instead of
this version parametrized by $A$ and $B$, but the proof is the same, adding
$\forall$ introductions on the root).

Define it as:

\begin{align*}
  map :=& map = \lambda f l n c . l n (\lambda v . c (f v))
\end{align*}

Note that actually $(\lambda v . c \:(f\: v))$ can be sugarized as $c \circ f$.
We could add a typing rule or a lemma to type compositions, but since it is
only used here we prefer to unfold to the pointwise term when used.




\begin{scprooftree}{0.86}
  \def\fCenter{\vdash}

  \AxiomC{}
  \RightLabel{Ax}
  \UnaryInfC{$\Gamma \fCenter
  l : \forall X . X \Arr (A \Arr X \Arr X) \Arr X) $}
  \RightLabel{$\forall_e$}
  \UnaryInfC{$\Gamma \fCenter l : X\Arr (A\Arr X\Arr X)\Arr X$}
  
  \AxiomC{}\RightLabel{Ax}
  \UnaryInfC{$\Gamma \fCenter n : X $}
  \RightLabel{$\Arr_e$}
  \BinaryInfC{$\Gamma \fCenter l n : (A \Arr X \Arr X) \Arr X$}
  \AxiomC{}\RightLabel{Ax}
  \UnaryInfC{$\Gamma_2 \fCenter c : B\Arr X\Arr X$}
  \AxiomC{}\RightLabel{Ax}
  \UnaryInfC{$\Gamma_2 \fCenter f: A\Arr B$}
  \AxiomC{}\RightLabel{Ax}
  \UnaryInfC{$\Gamma_2 \fCenter v :A$}
  \RightLabel{$\Arr_e$}
  \BinaryInfC{$\Gamma_2 \fCenter f v : B$}%%%%
  \RightLabel{$\Arr_e$}
  \BinaryInfC{$ ( \Gamma_2 := ) \Gamma, v: A
    \fCenter c (f v) : X \Arr X$}
  \RightLabel{$\Arr_i$}
  \UnaryInfC{$ \Gamma
    \fCenter \lambda v . c (f v) : A\Arr X \Arr X$}
  \RightLabel{$\Arr_e$}
  \BinaryInfC{$
    (\Gamma :=)f : A\Arr B, l:A^{*},n:X,c:B\Arr X \Arr X
    \fCenter l n (c \circ f) : X$}

  \RightLabel{$\Arr_i,\Arr_i,\forall_i ,\Arr_i,\Arr_i $}

  \UnaryInfC{$\fCenter
    \lambda f l n c . l n (c \circ f) : (A\Arr B) \Arr A^{*} \Arr B^{*}$}
  
\end{scprooftree}

Note that in the $\forall_e$ we substitute the type $X$ ( a type in
the context) to put it instead of the variable $X$.


Finally It is easy to show that $map$ satisfies its specification:

\begin{align*}
  map\: f\: nil            &\succ^{*} nil \\
  map\: f\: (cons\: a\: l) &\succ^{*} cons\: (f\: a) (map\: f\: l) 
\end{align*}


\end{document}
