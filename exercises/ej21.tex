\section{Exercise 21}

In general, in simply typed $\lambda$-calculus
if we can derive a typing judgement $\Gamma \vdash t : A$
then we can derive it extending the context.

Also, if we can make a derivation in HOL,
expanding the context (of assumptions) we can also make the derivation
(just ignore the new axioms).

Let's be more precise:

\begin{lemma}
  If $\Gamma \vdash t : A$ and $x \notin Dom(\Gamma)$
  then $\Gamma, x:X \vdash t : A$
\end{lemma}
\begin{proof}
  This is trivial by induction on $t$. Since the typing derivation
  is syntax directed, for each shape of $t$ there is only one rule
  that could be applied in that case.
  \begin{itemize}
  \item
    If $t = x$ then the only possible derivation is
    \begin{prooftree}
      \AxiomC{ $\Gamma(x) = A$ }
      \UnaryInfC{$x : A$}
    \end{prooftree}

    and then the following derivation is well-formed:
    
    \begin{prooftree}
      \AxiomC{ $ (\Gamma,y : X) (x) = A$ }
      \UnaryInfC{ $x : A$ }
    \end{prooftree}

    since the extension of a function does not change the value in
    a point of the previous domain.
    
  \item
    If $t = \lambda x.b$ then the typing derivation must be like:
    
    \begin{prooftree}
      \AxiomC{ $\Gamma, x:A \vdash b : B $ }
      \UnaryInfC{$\Gamma \vdash  \lambda x. b : A \Rightarrow B$}
    \end{prooftree}

    Then the following derivation is well formed:
    \begin{prooftree}
      \AxiomC{ $\Gamma, x:A, y:X \vdash b : B $ }
      \UnaryInfC{$\Gamma, y:X \vdash  \lambda x. b : A \Rightarrow B$}
    \end{prooftree}

  \item
    The application case is even more
    trivial since there is no manipulation of the context.
  \end{itemize}
\end{proof}



\begin{lemma}
  If $\Gamma \vdash t : A$ then $\Gamma,\Delta \vdash t : A$, assuming
  $\Gamma$ and $\Delta$ with disjoint domains.  
\end{lemma}

\begin{proof}
  This is trivial by induction in the size of $\Delta$. Note that we
  are assuming that contexts are finite.
\end{proof}

Then if $\gamma \vdash A : o $ we have
$\gamma,\delta \vdash A : o $. So if $\Gamma$ is a well formed
set of axioms under the context $\gamma$ of the underlying simply typed
lambda calculus, it is well formed under $\gamma, \delta$.

So $\gamma; \Gamma \vdash A$ implies $\gamma,\delta; \Gamma \vdash A$


.

.

.

.

.

.


This rule is not derivable. We are not doing sequent calculus so there
is no way to introduce formulas in the left. Formally this is proved
trivially by induction over the rules.

