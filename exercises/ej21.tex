\section{Exercise 21}

In general, in simply typed $\lambda$-calculus
if we can derive a typing judgement $\Gamma \vdash t : A$
then we can derive it extending the context, since when checking the context
the only interesting thing is if typing judgements belong to it
or not.

Also, if we can make a derivation in HOL,
expanding the context (of assumptions) we can also make the derivation,
just ignore the new axioms. (Weakening).

Let's be more precise:

\begin{lemma}
  
  If $\Gamma \vdash t : A$ and $x \notin Dom(\Gamma)$
  then $\Gamma, x:X \vdash t : A$
\end{lemma}
\begin{proof}
  This is straightforward by induction on $t$. Since the typing derivation
  is syntax directed, for each shape of $t$ there is only one rule
  that could be applied in that case.
  \begin{itemize}
  \item
    If $t = x$ then the only possible derivation is
    \begin{prooftree}
      \alwaysNoLine
      \AxiomC{$\vdots$}
      \UnaryInfC{ $\Gamma(x) = A$ }
      \alwaysSingleLine
      \UnaryInfC{$x : A$}
    \end{prooftree}

    and then the following derivation is well-formed:
    
    \begin{prooftree}
      \alwaysNoLine
      \AxiomC{$\vdots$}
      \UnaryInfC{ $ (\Gamma,y : T) (x) = A$ }
      \alwaysSingleLine
      \UnaryInfC{ $x : A$ }
    \end{prooftree}

    since the extension of a function does not change the value in
    a point of the previous domain.
    
  \item
    If $t = \lambda x.b$ then the typing derivation must be like:
    
    \begin{prooftree}
      \alwaysNoLine
      \AxiomC{\vdots}
      \UnaryInfC{ $\Gamma, x:A \vdash b : B $ }
      \alwaysSingleLine
      \UnaryInfC{$\Gamma \vdash  \lambda x. b : A \Rightarrow B$}
    \end{prooftree}

    Then the following derivation is well formed:
    \begin{prooftree}
      \alwaysNoLine
      \AxiomC{\vdots}
      \UnaryInfC{ $\Gamma, x:A, y:T \vdash b : B $ }
      \alwaysSingleLine
      \UnaryInfC{$\Gamma, y:T \vdash  \lambda x. b : A \Rightarrow B$}
    \end{prooftree}

    applying the induction
    hypothesis to the term $b$.
  \item
    The application case is even more
    trivial since there is no manipulation of the context.
    
    If $t = u v$ then the root of the typing proof is:

    \begin{prooftree}
      \alwaysNoLine
      \AxiomC{\vdots}
      \UnaryInfC{ $\Gamma \vdash a : A\Rightarrow B$ }
      \AxiomC{\vdots}
      \UnaryInfC{ $\Gamma \vdash b : A $ }
      \alwaysSingleLine
      \BinaryInfC{$\Gamma \vdash  a \, b : B$}
    \end{prooftree}
\newpage
    Then the following derivation is well formed: 
    
    
    \begin{prooftree}
      \alwaysNoLine
      \AxiomC{\vdots}
      \UnaryInfC{ $\Gamma, x : T \vdash a : A \Rightarrow B$ }
      \AxiomC{\vdots}
      \UnaryInfC{ $\Gamma, x : T \vdash b : A $ }
      \alwaysSingleLine
      \BinaryInfC{$\Gamma \vdash  a \, b : B$}
    \end{prooftree}

    applying induction hypothesis to $a$ and $b$.
  \end{itemize}
\end{proof}


\begin{lemma} [Weakening in simply typed $\lambda$-calculus]
  
  If $\Gamma \vdash t : A$ and $\Delta$ is a context with disjoint domain than
  $\Gamma$
  then $\Gamma, \Delta \vdash t : A$
\end{lemma}
\begin{proof}
  Trivial by induction on the length of $\Delta$, applying the previous lemma.
  Note that $\Gamma$ is finite, so we can do this.
\end{proof}


\begin{lemma} [Weakening in HOL]
\label{weakening}
  
  If $\gamma ; \Gamma \vdash t : A$ is derivable
  then $ \gamma ; \Gamma,\Delta \vdash t : A$ is derivable, assuming
  $\Gamma$ and $\Delta$ with disjoint domains.  
\end{lemma}

\begin{proof}
  I'll be more sloppy here since otherwise this is very verbose.

  This lemma is trivial by induction in the size of $\Delta$ once we prove
  that we can add a proposition to a context. (Note again that we
  are assuming that contexts are finite).

  To prove this, we can use induction over the rules. For example for
  the Axiom rule
  we only check that $A \in \Gamma$, so this is true when we extend
  $\Gamma$.

  For inductive rules in every one either we do not touch
  the context or we check if something belongs to it, which is conservative
  when we extend it.
\end{proof}

\begin{theorem}[Exercise 21]
  If $\: \gamma ; \Gamma \vdash A$ is derivable then $\: \gamma, \delta ;
  \Gamma, \Delta \vdash A$ is derivable.
\end{theorem}

If $\: \gamma \vdash A : o \:$ is derivable then we have
$\:\gamma,\delta \vdash A : o \:$
derivable by weakening in $\lambda$-calculus.
So if $\Gamma$ is a well formed
set of axioms under the context $\gamma$ of the underlying simply typed
lambda calculus, it is well formed under $\gamma, \delta$.

So $\gamma; \Gamma \vdash A$ derivable implies
$\gamma,\delta; \Gamma \vdash A$ derivable.
Applying weakening for HOL then $\gamma,\delta; \Gamma, \Delta \vdash A$ can
be derived, as we wanted.

This rule is not derivable. We are not doing sequent calculus so there
is no way to introduce formulas in the left. Formally this is proved
trivially by induction over the rules.
Note that when applying a rule, the
context is decreasing, by induction this is true for a proof,
which contradicts the possibility of derive this rule (At least if $\delta$
and $\Delta$ are nonempty).

