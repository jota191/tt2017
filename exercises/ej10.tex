
\section{Exercise 10 - Church-Rosser Theorem}

We must show that $\succ^{*}$ is confluent.

Define $D : \Lambda \rightarrow \Lambda$ as:
\begin{align*}
 & D(x) = x \\
 & D(\lambda x.t) = \lambda x. D(t)\\
 & D((\lambda x.t)u) = D(t)[x:=D(u)]\\
 & D(tu) = D(t) D(u) \text{\; if t is not an abstraction}
\end{align*}



\begin{lemma}
  If $t \succ^{*} t'$ and $u \succ^{*} u'$ then $t[x:=u] \succ^{*} t'[x:=u']$.
\end{lemma}
\begin{proof}
  Well known, it follows by induction over $t$ using the substitution lemma.
\end{proof}

We prove the following lemmas by induction over a term. We separate the
application in cases depending if the first term is an abstraction or not,
since the definition of $D$ is different in each case.

\begin{lemma}
 $ t \succ^{*} D(t)$
\end{lemma}

\begin{proof}
  By induction over $t$.

  \begin{itemize}
  \item
    If $t = x$ this is trivial since $D(t) = x$, so $t \succ^{0} D(t)$.
  \item
    If $t = \lambda x. a$, by induction hypothesis $a \succ^{*} D(a)$, since
    $a$ is smaller than $t$,
    so $\lambda x. a \succ^{*} \lambda x . D(a) = D(\lambda x. a)$
  \item
    If $t = (\lambda x . a) b$, then
    $(\lambda x . a) b \succ a[ x:= b] \succ^{*}_{hip} D(a)[x:=D(b)]
    = D((\lambda x.a) b)$ using Lemma 1 and induction hypothesis.
  \item
    Finally if $t = a b$ where $a$ is not an abstraction,
    \begin{flalign*}
      & a b &\succ^{*} &\text{ \{ by induction hypothesis, twice \} }
      \hskip0.7\textwidth\\
      & D(a)D(b) &=  &\text{ \{ by definition of $D$ \}}\\
      &D(a b)
    \end{flalign*}
  \end{itemize}
\end{proof}


\begin{lemma}
 If $ t \succ t'$ then $t' \succ^{*} D(t)$
\end{lemma}

\begin{proof}
  By induction over $t$ (considering the rules for beta reductions).
  \begin{itemize}
  \item
    If $t = x$ trivial (actually not possible, since $t$ has no redex).
  \item
    If $t= \lambda x . a$, then necessarly $t' = \lambda x. a'$
    where $a \succ a'$ (there is only one redex rule for abstraction,
    we reason by inversion). Then:
    \begin{flalign*}
      & t' &= & \\
      & \lambda x . a' &\succ^{*} &\text{\{ by induction hypothesis in $a$\}}
      \hskip0.7\textwidth\\
      & \lambda x . D(a) &=  &\text{\{ by definition of $D$\}} \\
      & D(\lambda x.a ) = D(t)
    \end{flalign*}
    \item
      If $t = (\lambda x . a) b$ it can reduce in one step to
      $(\lambda x . a') b$,
      $(\lambda x . a) b'$ or $a[x:= b]$ (where $a \succ a'$, $b \succ b'$).
      Applying one more reduction
      to the first two cases (in the outhermost redex), it is enough to show
      that $a'[x:= b]$, $a[x:= b']$, $a[x:= b]$ all reduces (in $*$ steps) to
      $D(t) = D(a) [x:= D(b)]$.
      This is trivial applying Lemma 1, and Lemma 2 and induction hypothesis
      to the instances of $t$ and $u$
      (induction to the primes, lemma to the untouched substerms).
      For instance $a[x:=b'] \succ^{*} D(a)[x:=D(b)]$ follows from applying
      Lemma 2 to a, inductive hypothesis to b', and Lemma 1 outside.
    \item
      If $t = a b$ with $a$ not being an abstraction, $t$ reduces to either
      $a' b$ or $a b'$ (where $a \succ a'$, $b \succ b'$). $D(t) = D(a)D(b)$,
      Apply Lemma 2 and induction hypothesis to the suitable subterm again,
      and we are done.
  \end{itemize}
\end{proof}

\begin{lemma}
  It $t \succ u$ then $D(t) \succ^{*} D(u)$
\end{lemma}
\begin{proof}
  Again, by induction over $t$, considering the possible beta reductions.

  \begin{itemize}
  \item
    $t=x$ is trivial, no reduction.
  \item
    $t= \lambda x. a$ can only reduce to $u = \lambda x. a'$
    with $a \succ a'$,
    then $D(t) = \lambda x. D(a)$ and $D(u) = \lambda x . D(a')$, from
    induction hypothesis ($a$ is smaller than $t$) $D(a) \succ^{*} D(a')$ and
    the result follows.
  \item
    If $t = a b$ where $a$ is not an abstraction, it is inmediate again
    applying the induction hypothesis in the term that is reduced
    (this is allways the easy -inductive- case).
  \item
    If $t = (\lambda x. a) b$, $t$ can reduce to $(\lambda x . a') b$,
    $(\lambda x . a) b'$ or $a[x:= b]$ (where $a \succ a'$, $b \succ b'$).
    In the last two cases it is easy to apply the induction hypothesis,
    in the former one, when $t$ reduces to $ u = a[x:= b]$, so we must
    show that $D(t) = D((\lambda x . a) b) = D(a)[x := D(b)]$ reduces
    (in $*$ steps) to
    $D(a[x:= b])$. We demonstrate this in the next lemma.
  \end{itemize}
\end{proof}

\begin{lemma}
  $D(t)[x:= D(u)] \succ^{*} D(t[x:=u])$
\end{lemma}

\begin{proof}
  By induction over $t$.
  \begin{itemize}
  \item
    If $t = y$, again it is trivial. If y = x both sides are $D(u)$, if
    $y \neq x$ both sides are $D(y)$.
  \item
    if $t= \lambda y . a$ then
    \begin{flalign*}
      &D(t)[x:=D(u)]  &= &\\
      &D(\lambda y.a)[x:=D(u)] &= &\text{ \{ by definition of $D$\} }
      \hskip0.7\textwidth\\
      &(\lambda y.D(a))[x:=D(u)]&= &\text{\{by definition of substitution,
        assuming $y \neq x$, otherwise rename y \}}\\
      &\lambda y.(D(a)[x:=D(u)])&\succ^{*}
      &\text{\{by inductive hypothesis\}}\\
      &\lambda y.D(a[x:=u])&= &\text{\{by definition of $D$\}}\\
      &D(\lambda y.(a[x:=u]))&= &\text{\{by definition of substitution\}}\\
      &D((\lambda y.a)[x:=u])&= &\\
      &D(t[x:=u])& &\\
    \end{flalign*}
  \item
    The application case is tricky. We must study separate cases, one more
    than before, We will see why.
    \begin{itemize}
    \item
      If $t = (\lambda y . a) b$:
      \begin{flalign*}
        &D(t)[x:=D(u)]  &= &\\
        &D((\lambda y.a)b)[x:=D(u)] &= &\text{ \{ by definition of $D$\} }
        \hskip0.7\textwidth\\
        &(D(a)[y := D(b)])[x:=D(u)] &\succ^{*} &
        \text{ \{ by substitution lemma\} }\\
      &(D(a)[x := D(u)])[y := D(b)[x:=D(u)]] &\succ^{*}
      &\text{ \{ by inductive hypothesis \} }\\
      &(D(a[x := u)])[y := D(b)[x:=D(u)]] &\succ^{*}
      &\text{ by inductive hypothesis}\\
      &(D(a[x := u)])[y := D(b[x:=u]] &=
      &\text{ by definition of $D$}\\
      &D((\lambda y. a[x:= u])(b[x:=u])) &=
      &\text{ by definition of substitution}\\
      &D(((\lambda y. a)[x:= u])(b[x:=u])) &=
      &\text{ by definition of substitution}\\
      &D((\lambda y. a)b[x:= u]) &=
      &\\
      &D(t[x:= u]) &
      &
      \end{flalign*}
    \item
      If $t = a b$, and a is not a variable, then:
      \begin{flalign*}
        &D(t)[x:=D(u)]  &= &\\
        &D(a b)[x:=D(u)]&= &\text{ \{ by definition of $D$\} }
        \hskip0.7\textwidth\\
        &(D(a) D(b))[x:=D(u)]&= &\text{ \{ by definition of substitution\}}\\
        &(D(a)[x:=D(u)]) (D(b)[x:=D(u)])&= &\text{\{by hypothesis, twice\}}\\
        &D(a[x:=u]) D(b[x:=u])&= &\text{ \{ by definition of $D$\} }\\
        &D(a[x:=u] b[x:=u])&= &\text{ \{ by definition of substitution\} }\\
        &D((a b)[x:=u])&= &\\
        &D(t[x:=u])& &
      \end{flalign*}
      \item
        If $t = x b$
        this case is different to the previous one since the substitution
        could generate a new redex. Note that if $t = y b$
        where $y \neq x$ the proof is like in the former case.
      \begin{flalign*}
        &D(t)[x:=D(u)]  &= &\\
        &D(x b)[x:=D(u)]&= &\text{ \{ by definition of $D$\} }
        \hskip0.7\textwidth\\
        &(D(x) D(b))[x:=D(u)]&= &\text{ \{ by definition of substitution\}}\\
        &(D(x)[x:=D(u)]) (D(b)[x:=D(u)])&=
        &\text{ \{ by definition of substitution and $D$ \} }\\
        &(D(u)) (D(b)[x:=D(u)])&=
        &\text{ \{ applying inductive hypothesis\} }\\
        &(D(u)) (D(b[x:=u]))&=
        &\text{ \{ applying substitution backwards\} }
      \end{flalign*}
      If $u$ is not an abstraction there is no problem (rewrite
      the definition of $D$ backwards and so on). If $u = \lambda y.c$ then:
      \begin{flalign*}
        &(D(u)) (D(b[x:=u]))&=
        &
        \hskip0.7\textwidth\\
        &(D(\lambda y.c)) (D(b[x:=u]))&=
        &\text{ \{ by definition of $D$\} }\\
        &(\lambda y.D(c)) (D(b[x:=u]))&\succ
        &\text{ }\\
        &D(c)[y:= (D(b[x:=u]))]&=
        &\text{ \{ rewriting $D$\}}\\
        &D((\lambda y.c)(b[x:=u])) &= &\\
          &D(u (b[x:=u]))&= & \text{ \{ by definition of substitution\}}\\
            &D((x [x:= u]) (b[x:=u]))&= &
              \text{ \{ by definition of substitution\}}\\
              &D(x u [x:=u])& &
      \end{flalign*}
    \end{itemize}
  \end{itemize}
\end{proof}

\begin{lemma}
  $t \succ^{*} u$ implies
  $D(t) \succ^{*} D(u)$
\end{lemma}
\begin{proof}
  By induction over the number of steps of the $t \succ^{*} u$ reduction.
  If $n = 1$ then this is an instance of Lemma 4.

  If $t\succ^{n} u$ then consider the first reduction, so $t \succ t'$,
  $t' \succ^{n-1} u$. By induction hypothesis $D(t') \succ^{*} D(u)$,
  and by Lemma 4 $D(t) \succ^{*} D(t')$. By transitivity of $\succ^{*}$
  follows that $D(t) \succ^{*} D(u)$.
\end{proof}


\begin{lemma}
  $t \succ^{n} u$ implies
  $u \succ^{*} D^{n}(t)$
\end{lemma}
\begin{proof}
  Again, by induction over the length of reduction.

  For the base case, this is exactly Lemma 3.

  For $t \succ^{n} u$ with $n>1$ consider again the first reduction,
  so $t \succ t'$ and $t' \succ^{n-1} u$. $t' \succ^{*} D(t)$ by Lemma 3, and
  by induction hypothesis $u \succ^{*} D^{n-1}(t')$, Applying Lemma 6
  $D^{n-1}(t') \succ^{*} D^{n-1}(D(t)) = D^{n}(t)$. By transitivity of
  $\succ^{*}$ then $u \succ^{*} D^{n}(t)$ as we claim.
\end{proof}

\begin{theorem}[Church-Rosser Theorem]
  The relation $\succ^{*}$ is confluent. This is, if $t \succ^{*} t_1$
  and $t \succ^{*} t_2$ then there is a term $v$ such that $t_1 \succ^{*} v$
  and $t_2 \succ^{*} v$.
\end{theorem}

\begin{proof}
  If $t \succ^{n} t_1$ and $t \succ^{m} t_2$, then applying Lemma 7
  $t_1 \succ^{*} D^n(t)$ and $t_2 \succ^{*} D^m(t)$.
  Without loss of generality assume that $m \geq n$. It is easy to show that
  $D^n(t) \succ^{*} D^m(t)$ (Applying Lemma 2 $m-n$ times,
  $t \succ^{*} D^{m-n}(t)$, then Apply Lemma 6 $n$ times, or more formal, we
  can prove by an easy induction).

  Then by transitivity of $\succ^{*}$ is true that $t_1 \succ^{*} D^m(t)$,
  take $u = D^m(t)$.
\end{proof}
